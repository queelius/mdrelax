%% ============================================================================
\section{Conclusion}
\label{sec:conclusion}
%% ============================================================================

We have studied the sensitivity of maximum likelihood estimation in masked
series systems to violations of the non-informative masking assumption (C2).
Three contributions emerge:

\begin{enumerate}
    \item \textbf{Misspecification characterization.} When C2 is violated but
        assumed to hold, the MLE converges to a pseudo-true parameter whose
        individual component parameters absorb the masking asymmetry, while the
        system hazard remains consistently estimated
        (\Cref{thm:misspec-C2}). Joint estimation of the masking mechanism
        and the reliability parameters is not possible due to fundamental
        non-identifiability (\Cref{thm:non-ident}).

    \item \textbf{Quantitative sensitivity map.} A systematic simulation sweep
        across 11 severity levels for a 5-component Weibull series system
        reveals that system-level quantities (system hazard) are robust to C2
        violations across the full range (bias $< 3\%$), while individual
        component estimates degrade with severity. The exponential
        component within the system shows notable resistance to
        misspecification.

    \item \textbf{Practical guidance.} Since the masking mechanism is
        unknowable from the data alone, sensitivity analysis over plausible
        masking structures is the appropriate response. For system-level
        inference, the standard C1-C2-C3 model suffices. For component-level
        inference, we recommend reporting results under multiple masking
        scenarios.
\end{enumerate}

All methods are implemented in the \texttt{mdrelax} R package, available at
\url{https://github.com/queelius/mdrelax}.
