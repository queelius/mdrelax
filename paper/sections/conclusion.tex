%% ============================================================================
\section{Conclusion}
\label{sec:conclusion}
%% ============================================================================

We have developed a theoretical framework for likelihood-based inference in
series systems with masked failure data when the traditional conditions C2
(non-informative masking) and C3 (parameter-independent masking) are relaxed.
Our main contributions are:

\begin{enumerate}
    \item \textbf{Likelihood derivations.} We established the form of the
        likelihood under various relaxation scenarios, showing that the
        masking probabilities act as weights on component hazard contributions
        when C2 is violated.

    \item \textbf{Practical masking models.} We introduced rank-based
        informative masking and KL-divergence constrained models that provide
        interpretable parameterizations of non-standard masking.

    \item \textbf{Identifiability results.} We proved that informative masking
        can paradoxically \emph{improve} identifiability by breaking symmetries
        that cause non-identifiability under standard conditions.

    \item \textbf{Fisher information analysis.} We derived closed-form
        expressions for the Fisher information matrix under informative
        masking for exponential series systems, enabling efficiency comparisons.

    \item \textbf{Misspecification analysis.} We characterized the bias that
        arises from incorrectly assuming C2 or C3 when masking is actually
        informative or parameter-dependent, providing guidance on when
        relaxed models are necessary.

    \item \textbf{Weibull extension.} We demonstrated that the framework
        extends naturally to Weibull components, with simulation studies
        confirming the robustness findings observed for exponential systems
        while identifying additional challenges from shape-scale interactions.
\end{enumerate}

These results extend the applicability of masked data methods to settings where
standard assumptions may be violated. The accompanying \texttt{mdrelax} R package
provides implementation of these methods for practitioners. Directions for
future research are discussed in \Cref{sec:future}.
