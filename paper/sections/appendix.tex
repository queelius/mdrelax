%% ============================================================================
\section{Proofs and Derivations}
\label{app:proofs}
%% ============================================================================

\subsection{Score Function Under Informative Masking}
\label{app:score-inform}

For completeness, we provide the full derivation of the score function under
the Bernoulli informative masking model for exponential series systems.

Let the log-likelihood contribution from an uncensored observation be:
\begin{equation}
    \ell_i(\vlambda) = -s_i \sum_{j=1}^m \lambda_j +
    \log\left(\sum_{k \in c_i} \lambda_k \pi_{k,c_i}\right).
\end{equation}

The partial derivative with respect to $\lambda_j$ is:
\begin{align}
    \frac{\partial \ell_i}{\partial \lambda_j}
    &= -s_i + \frac{\partial}{\partial \lambda_j}
        \log\left(\sum_{k \in c_i} \lambda_k \pi_{k,c_i}\right) \\
    &= -s_i + \frac{\pi_{j,c_i} \ind{j \in c_i}}
                   {\sum_{k \in c_i} \lambda_k \pi_{k,c_i}}.
\end{align}

The total score is:
\begin{equation}
    \frac{\partial \ell}{\partial \lambda_j} =
    \sum_{i=1}^n \frac{\partial \ell_i}{\partial \lambda_j} =
    -\sum_{i=1}^n s_i + \sum_{i: \delta_i = 1}
    \frac{\pi_{j,c_i} \ind{j \in c_i}}
         {\sum_{k \in c_i} \lambda_k \pi_{k,c_i}}.
\end{equation}

Setting this to zero and solving gives the MLE equations under informative
masking.

\subsection{Hessian Matrix Derivation}
\label{app:hessian}

The second partial derivatives are:
\begin{align}
    \frac{\partial^2 \ell_i}{\partial \lambda_j \partial \lambda_\ell}
    &= \frac{\partial}{\partial \lambda_\ell} \left[
        \frac{\pi_{j,c_i} \ind{j \in c_i}}
             {\sum_{k \in c_i} \lambda_k \pi_{k,c_i}}
    \right] \\
    &= -\frac{\pi_{j,c_i} \pi_{\ell,c_i} \ind{j \in c_i} \ind{\ell \in c_i}}
             {\left(\sum_{k \in c_i} \lambda_k \pi_{k,c_i}\right)^2}.
\end{align}

The observed Fisher information matrix is the negative Hessian:
\begin{equation}
    \FIM_{j\ell}(\vlambda) = -\frac{\partial^2 \ell}{\partial \lambda_j \partial \lambda_\ell}
    = \sum_{i: \delta_i = 1}
    \frac{\pi_{j,c_i} \pi_{\ell,c_i} \ind{j \in c_i} \ind{\ell \in c_i}}
         {\left(\sum_{k \in c_i} \lambda_k \pi_{k,c_i}\right)^2}.
\end{equation}

\subsection{Expected Fisher Information}
\label{app:expected-fim}

The expected FIM requires integrating over the distribution of candidate sets.
For the exponential series system under C1-C2-C3 with Bernoulli masking
(each non-failed component in candidate set with probability $p$), the
expected Fisher information per observation is:
\begin{equation}
    \E[\FIM_{jk}] = \E\left[\frac{\ind{j \in C} \ind{k \in C}}
    {\left(\sum_{\ell \in C} \lambda_\ell\right)^2}\right],
\end{equation}
where the expectation is over both $K$ (failed component) and $C$
(candidate set).

This can be expanded as:
\begin{align}
    \E[\FIM_{jk}] &= \sum_{k_0=1}^m \Prob\{K = k_0\}
    \E\left[\frac{\ind{j \in C} \ind{k \in C}}
        {\left(\sum_{\ell \in C} \lambda_\ell\right)^2} \Bigg| K = k_0\right].
\end{align}

Under C1, the failed component $k_0$ is always in $C$. The expectation
over candidate sets involves summing over all possible $C \ni k_0$
weighted by their probabilities under the Bernoulli model:
\begin{equation}
    \Prob\{C = c \mid K = k_0\} = \prod_{j \in c \setminus \{k_0\}} p
    \prod_{j \notin c} (1-p).
\end{equation}

Closed-form evaluation of this expectation is generally intractable due to
the sum in the denominator. Monte Carlo estimation or numerical integration
is typically required.

%% ============================================================================
\section{Implementation Details}
\label{app:implementation}
%% ============================================================================

\subsection{R Package Functions}
\label{app:functions}

The theoretical framework developed in this paper is implemented in the
\texttt{mdrelax} R package. Key
functions include:

\begin{itemize}
    \item \texttt{md\_bernoulli\_cand\_C1\_C2\_C3()}: Generates candidate set
        probabilities under the standard Bernoulli model satisfying C1-C2-C3.

    \item \texttt{md\_bernoulli\_cand\_C1\_kld()}: Generates candidate set
        probabilities with a specified KL-divergence from the baseline
        C1-C2-C3 model.

    \item \texttt{informative\_masking\_by\_rank()}: Computes inclusion
        probabilities based on component failure time ranks.

    \item \texttt{md\_cand\_sampler()}: Samples candidate sets from
        probability vectors.

    \item \texttt{md\_loglike\_exp\_series\_C1\_C2\_C3()}: Log-likelihood
        function for exponential series systems under C1-C2-C3.

    \item \texttt{md\_mle\_exp\_series\_C1\_C2\_C3()}: Maximum likelihood
        estimation for exponential series systems.

    \item \texttt{md\_fim\_exp\_series\_C1\_C2\_C3()}: Observed Fisher
        information matrix for exponential series systems.

    \item \texttt{md\_block\_candidate\_m3()}: Demonstrates block
        non-identifiability in a 3-component system.
\end{itemize}

\subsection{Optimization Details}
\label{app:optimization}

MLE is computed using the L-BFGS-B algorithm \citep{byrd1995} with
analytically computed gradients. The optimization is initialized using
a method-of-moments estimator based on the total system hazard:
\begin{equation}
    \hat{\Lambda}_{\text{init}} = \frac{n_{\text{uncensored}}}{\sum_{i=1}^n s_i},
    \quad
    \hat{\lambda}_{j,\text{init}} = \frac{\hat{\Lambda}_{\text{init}}}{m}.
\end{equation}

For challenging optimization landscapes, simulated annealing can be used to
find a good starting point before local optimization.

%% ============================================================================
\section{Additional Simulation Results}
\label{app:sim-results}
%% ============================================================================

Additional simulation results supplement the findings in Section~\ref{sec:simulations}.
The complete results are available in the R package's simulation directory.

\begin{enumerate}
    \item Full tables of bias, RMSE, and coverage for all parameter
        configurations are provided in the package's \texttt{inst/simulations/results/}
        directory.
    \item The simulation scripts in \texttt{inst/simulations/} can be used to
        reproduce all results and generate diagnostic plots.
    \item Sensitivity analyses for misspecified masking parameters show
        that the C2 assumption is robust up to moderate departures, with
        RMSE ratios remaining below 1.10 across tested configurations.
    \item The simulation framework supports arbitrary component configurations
        and can be extended to Weibull series systems.
\end{enumerate}
