%% ============================================================================
\section{Simulation Study}
\label{sec:simulations}
%% ============================================================================

We conduct a systematic sweep of C2 violation severity for a 5-component
Weibull series system, fitting the standard C1-C2-C3 model throughout. This
quantifies the practical impact of C2 misspecification as a function of the
departure from non-informative masking.

\subsection{Design}
\label{sec:sim-design}

\paragraph{System configuration.}
We consider an $m = 5$ component Weibull series system with shapes
$\mathbf{k}^* = (2.0, 1.5, 1.2, 1.8, 1.0)$ and scales
$\vlambda^* = (3.0, 4.0, 5.0, 3.5, 4.5)$. The fifth component ($k_5 = 1.0$)
is exponential, providing a natural point of comparison within the system.
Right-censoring time is $\tau = 5$, sample size $n = 500$, and $B = 200$
replications per severity level. The true system hazard at reference time
$t_0 = 2.0$ is $h_T^*(2.0) = 1.460$.

\paragraph{Severity sweep.}
We sweep the severity parameter $s$ over $\{0, 0.1, 0.2, \ldots, 1.0\}$
(11 levels). At each level, data is generated from the Bernoulli model with
$\mathbf{P}(s)$ as defined in \eqref{eq:P-sweep}, using base probability
$p_0 = 0.5$. The direction matrix has a cyclic structure with rows summing
to zero:
\begin{equation}
\label{eq:D-matrix}
    \mathbf{D} = \begin{pmatrix}
         0   & 0.3  & -0.2 & 0.1  & -0.2 \\
        -0.2 & 0    & 0.3  & -0.2 & 0.1  \\
         0.1 & -0.2 & 0    & 0.3  & -0.2 \\
        -0.2 & 0.1  & -0.2 & 0    & 0.3  \\
         0.3 & -0.2 & 0.1  & -0.2 & 0
    \end{pmatrix}.
\end{equation}
The cyclic structure ensures that each component is equally affected in
aggregate, so the bias pattern is driven by the interaction between masking
asymmetry and the heterogeneous Weibull parameters rather than the direction
matrix itself.

\paragraph{Estimation.}
At each severity level, all $B$ datasets are fitted with the standard
\emph{C1-C2-C3} Weibull series MLE from \eqref{eq:loglike-wei}, ignoring the
informative masking. Optimization uses L-BFGS-B with starting values at 90\%
of true parameters. Convergence was 100\% across all conditions.

\paragraph{Metrics.}
For each parameter and severity level, we report:
\begin{itemize}
    \item \textbf{Bias}: $\bar{\hat\theta}_j - \theta_j^*$,
    \item \textbf{RMSE}: $\sqrt{B^{-1}\sum_b (\hat\theta_j^{(b)} - \theta_j^*)^2}$,
    \item \textbf{Coverage}: proportion of 95\% Wald confidence intervals
        $\hat\theta_j \pm 1.96\,\hat{\text{se}}_j$ that contain $\theta_j^*$.
\end{itemize}
We additionally track the system hazard $\hat{h}_T(t_0)$ evaluated at
$t_0 = 2.0$.

\subsection{Results}
\label{sec:sim-results}

\Cref{fig:bias,fig:rmse,fig:coverage} display the three metrics as a function
of severity.

\begin{figure}[htbp]
    \centering
    \includegraphics[width=0.85\textwidth]{fig_bias_vs_severity.pdf}
    \caption{Bias as a function of C2 violation severity for a 5-component
    Weibull series system. Individual component parameters show increasing bias
    with severity, while the system hazard (triangle markers) remains
    approximately unbiased.}
    \label{fig:bias}
\end{figure}

\begin{figure}[htbp]
    \centering
    \includegraphics[width=0.85\textwidth]{fig_rmse_vs_severity.pdf}
    \caption{Root mean squared error as a function of C2 violation severity.
    The RMSE for individual parameters grows with severity; the system hazard
    RMSE remains stable near 0.10 throughout.}
    \label{fig:rmse}
\end{figure}

\begin{figure}[htbp]
    \centering
    \includegraphics[width=0.85\textwidth]{fig_coverage_vs_severity.pdf}
    \caption{95\% confidence interval coverage as a function of C2 violation
    severity. The dashed line marks the nominal 95\% level. Coverage for the
    most affected shape parameters ($k_2$, $k_3$, $k_4$) drops below 70\%
    at $s = 1.0$.}
    \label{fig:coverage}
\end{figure}

\Cref{tab:wei-sweep} provides numerical summaries at selected severity levels
for a representative subset of parameters. The full results for all 10
component parameters are shown in the figures.

\begin{table}[htbp]
\centering
\caption{Weibull sensitivity sweep ($m=5$, $n=500$, $B=200$). Selected
parameters illustrating the range of misspecification effects. True system
hazard $h_T^*(2.0) = 1.460$.}
\label{tab:wei-sweep}
\small
\begin{tabular}{cl rrr}
\toprule
$s$ & Parameter & Bias & RMSE & Coverage \\
\midrule
0.0 & $k_2$ ($1.5$) & $+0.032$ & 0.259 & 0.960 \\
    & $\lambda_3$ ($5.0$) & $+0.231$ & 1.538 & 0.905 \\
    & $k_4$ ($1.8$) & $-0.001$ & 0.268 & 0.930 \\
    & $k_5$ ($1.0$) & $+0.018$ & 0.111 & 0.945 \\
    & $h_T(2.0)$ & $+0.024$ & 0.108 & --- \\
\addlinespace
0.2 & $k_2$ ($1.5$) & $-0.064$ & 0.251 & 0.905 \\
    & $\lambda_3$ ($5.0$) & $+0.366$ & 3.449 & --- \\
    & $k_4$ ($1.8$) & $+0.000$ & 0.303 & 0.930 \\
    & $k_5$ ($1.0$) & $+0.028$ & 0.125 & 0.930 \\
    & $h_T(2.0)$ & $+0.031$ & 0.105 & --- \\
\addlinespace
0.5 & $k_2$ ($1.5$) & $-0.194$ & 0.305 & 0.790 \\
    & $\lambda_3$ ($5.0$) & $-0.287$ & 1.465 & 0.800 \\
    & $k_4$ ($1.8$) & $-0.117$ & 0.322 & 0.875 \\
    & $k_5$ ($1.0$) & $+0.027$ & 0.129 & 0.940 \\
    & $h_T(2.0)$ & $+0.028$ & 0.108 & --- \\
\addlinespace
0.8 & $k_2$ ($1.5$) & $-0.302$ & 0.346 & 0.540 \\
    & $\lambda_3$ ($5.0$) & $-0.596$ & 1.720 & 0.670 \\
    & $k_4$ ($1.8$) & $-0.251$ & 0.378 & 0.730 \\
    & $k_5$ ($1.0$) & $+0.080$ & 0.155 & 0.935 \\
    & $h_T(2.0)$ & $+0.012$ & 0.103 & --- \\
\addlinespace
1.0 & $k_2$ ($1.5$) & $-0.338$ & 0.384 & --- \\
    & $\lambda_3$ ($5.0$) & $-0.712$ & 1.385 & 0.665 \\
    & $k_4$ ($1.8$) & $-0.315$ & 0.401 & 0.650 \\
    & $k_5$ ($1.0$) & $+0.084$ & 0.160 & 0.930 \\
    & $h_T(2.0)$ & $+0.038$ & 0.109 & --- \\
\bottomrule
\end{tabular}
\end{table}

\subsection{Interpretation}
\label{sec:sim-interpret}

\paragraph{System-level robustness.}
The most striking finding is the stability of the system hazard. Across all
severity levels from $s = 0$ to $s = 1.0$, the system hazard bias remains
between $+0.012$ and $+0.038$ (at most 2.6\% relative error), and the RMSE
is virtually constant near 0.10. This confirms part~(a) of
\Cref{thm:misspec-C2}: the misspecified estimator preserves the system hazard
even when individual component parameters are severely biased. The robustness
holds despite the system having 10 free parameters (five shape-scale pairs)
and substantial heterogeneity across components.

\paragraph{Component-level degradation.}
Individual parameter estimates degrade with severity. The shape parameter
$k_2$ (true value 1.5) shows bias growing from $+0.03$ at $s = 0$ to $-0.34$
at $s = 1.0$ (23\% relative), with coverage dropping from 0.96 to below
nominal. The shape parameters $k_3$ and $k_4$ show similar degradation, with
coverage falling to 0.70 and 0.65 respectively at full severity.

\paragraph{The exponential component.}
Component~5 ($k_5 = 1.0$, exponential) is notably resistant to
misspecification: its shape bias reaches only $+0.08$ at $s = 1.0$ (8\%
relative) with coverage of 0.93, near the nominal level. This suggests that
the exponential case is a natural ``robustness baseline'' within the Weibull
family, consistent with the theoretical analysis in \Cref{sec:misspec} where
the exponential score is simpler and less susceptible to masking weight
distortion.

\paragraph{The breakdown boundary.}
For practitioners concerned with individual component estimates, the simulation
results suggest a rough breakdown boundary around $s \approx 0.3$: below this
threshold, component-level biases remain within typical estimation uncertainty
and coverage stays above 90\%. Above this threshold, the standard model
increasingly misattributes masking effects to component failure rates. For
system-level inference (system hazard, mean time to failure), the standard
model remains reliable across the full severity range.
