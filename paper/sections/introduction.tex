%% ============================================================================
\section{Introduction}
\label{sec:introduction}
%% ============================================================================

Estimating the reliability of individual components in a series system is a
fundamental challenge when failure causes are masked. When a series system
fails, diagnostic procedures may identify only a \emph{candidate set} of
components that could have caused the failure, rather than pinpointing the
exact failed component \citep{Usher-1988}. Maximum likelihood estimation under
masked data has been developed extensively under a standard set of
assumptions~\citep{Lin-1993, towell2023reliability}: (C1)~the true failed
component is always in the candidate set; (C2)~masking is non-informative,
meaning the candidate set formation is independent of the failure cause given
the failure time; and (C3)~the masking mechanism does not depend on the
component reliability parameters.

Among these, C2 is the strongest and least verifiable assumption. It requires
that the diagnostic process which generates candidate sets reveals nothing
about which component actually failed beyond the guarantee of C1. In practice,
the masking mechanism reflects a complex diagnostic process---technician
experience, test equipment sensitivity, failure mode characteristics---that is
rarely known and seldom non-informative. An experienced technician may
preferentially include components whose failure signatures match the observed
failure time; an automated diagnostic may weight components by their estimated
reliability ranking, effectively violating C2.

Prior work has addressed dependent masking by proposing alternative models for
the masking probability. \citet{LinGuess1994} introduced proportional masking
probabilities for two-component systems with closed-form maximum likelihood
estimators. \citet{Guttman1995} developed Bayesian inference under dependent
masking for two-component systems. \citet{Mukhopadhyay2006} extended
likelihood-based inference to general $m$-component systems where masking
depends on the cause of failure. \citet{CraiuReiser2010} studied conditional
masking probability models and their identifiability. These approaches share a
common structure: they propose a specific masking model to replace C2 and
estimate its parameters jointly with the component reliability parameters.

We take a complementary approach. Rather than proposing an alternative masking
model, we study the \emph{sensitivity} of the standard C1-C2-C3 estimator to
violations of C2. The question is practical: when a reliability engineer fits
the standard model---as is common practice---how wrong are the resulting
estimates when C2 fails? Our contributions are:
\begin{enumerate}
    \item \textbf{Misspecification characterization.} We prove that when C2 is
        violated but assumed to hold, the MLE converges to a pseudo-true
        parameter whose individual component rates absorb the masking
        asymmetry. The total system hazard, however, remains consistently
        estimated (\Cref{sec:framework}).

    \item \textbf{Simulation sweep.} Using a Bernoulli perturbation model that
        generates controlled C2 violations of varying severity, we map the
        bias, RMSE, and coverage degradation as a function of violation
        severity for a 5-component Weibull series system
        (\Cref{sec:simulations}).

    \item \textbf{Practical finding.} System-level reliability estimates
        (total hazard, mean time to failure) are robust to C2 violations
        across the full severity range. Individual component parameter
        estimates are not. Since the masking mechanism and component rates are
        jointly non-identifiable, sensitivity analysis---not model
        elaboration---is the appropriate response (\Cref{sec:discussion}).
\end{enumerate}

The remainder of this paper is organized as follows. \Cref{sec:background}
reviews series systems, masked data, and the standard C1-C2-C3 likelihood, and
surveys prior work on dependent masking. \Cref{sec:framework} develops the
sensitivity framework: the general likelihood under C1 alone, the Bernoulli
perturbation model, the misspecification theorem, and the identifiability
result. \Cref{sec:simulations} presents the simulation sweep.
\Cref{sec:discussion} discusses practical implications, and
\Cref{sec:conclusion} concludes.
