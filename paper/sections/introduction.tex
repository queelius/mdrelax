%% ============================================================================
\section{Introduction}
\label{sec:introduction}
%% ============================================================================

Estimating the reliability of individual components in a series system presents
a fundamental challenge in reliability engineering: system-level failure data
is observable, but component-level failure causes are often masked. When a
series system fails, diagnostic procedures may identify only a \emph{candidate
set} of components that could have caused the failure, rather than pinpointing
the exact failed component \citep{Usher-1988}. This masking, combined with
right-censoring of system lifetimes, complicates statistical inference about
component reliability parameters.

Prior work on masked data in series systems has established a tractable
likelihood framework under three conditions \citep{towell2023reliability}:
\begin{itemize}
    \item \textbf{C1:} The failed component is always contained in the
        candidate set.
    \item \textbf{C2:} Masking is non-informative---conditional on the failure
        time and candidate set, each component in the candidate set is equally
        likely to have failed.
    \item \textbf{C3:} The masking mechanism does not depend on the system
        parameters $\vtheta$.
\end{itemize}
Under these conditions, the likelihood has a simple closed form that enables
efficient maximum likelihood estimation (MLE). However, practical diagnostic
systems may violate C2 and C3:
\begin{itemize}
    \item Experienced technicians may preferentially include components that
        ``seem likely'' to have failed based on failure time characteristics,
        violating C2.
    \item Diagnostic algorithms based on reliability rankings may systematically
        favor certain components, with the ranking depending on the true
        parameters, violating C3.
\end{itemize}

This paper develops the theoretical and computational framework for
likelihood-based inference when C2 and/or C3 are relaxed while maintaining C1.
Our contributions are:
\begin{enumerate}
    \item \textbf{General likelihood framework.} We derive the likelihood under
        C1 alone, showing how informative and parameter-dependent masking
        modify the standard likelihood structure (\Cref{sec:relaxed}).

    \item \textbf{Practical masking models.} We introduce the rank-based
        informative masking model and the Bernoulli candidate set model with
        KL-divergence constraints, which provide interpretable parameterizations
        of non-standard masking (\Cref{sec:relaxed}).

    \item \textbf{Identifiability analysis.} We establish conditions under which
        parameters remain identifiable when standard conditions fail, including
        the surprising result that informative masking can \emph{improve}
        identifiability in certain cases (\Cref{sec:identifiability}).

    \item \textbf{Efficiency comparison.} We derive the Fisher information
        matrix under relaxed conditions for exponential series systems,
        enabling precise comparison of estimation efficiency
        (\Cref{sec:identifiability}).

    \item \textbf{Simulation studies.} We quantify the bias from incorrectly
        assuming C2 when masking is informative, and demonstrate the improved
        estimation achievable when the masking structure is properly modeled
        (\Cref{sec:simulations}).
\end{enumerate}

The remainder of this paper is organized as follows. \Cref{sec:background}
reviews series systems, masked data, and the standard C1-C2-C3 likelihood.
\Cref{sec:relaxed} develops the likelihood under relaxed conditions.
\Cref{sec:identifiability} analyzes identifiability and Fisher information.
\Cref{sec:simulations} presents simulation studies. \Cref{sec:discussion}
discusses practical implications, and \Cref{sec:conclusion} concludes.
