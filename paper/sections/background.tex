%% ============================================================================
\section{Background}
\label{sec:background}
%% ============================================================================

\subsection{Series System Model}
\label{sec:series-model}

Consider a series system composed of $m$ components. The lifetime of the
$i$-th system is
\begin{equation}
    T_i = \min\{T_{i1}, T_{i2}, \ldots, T_{im}\},
\end{equation}
where $T_{ij}$ denotes the lifetime of the $j$-th component in the $i$-th
system. Component lifetimes are assumed independent with parametric
distributions indexed by $\vtheta_j$; the full parameter vector is
$\vtheta = (\vtheta_1, \ldots, \vtheta_m) \in \vOmega$.

For component $j$ with parameter $\vtheta_j$, the reliability, density, and
hazard functions are
\begin{equation}
    R_j(t; \vtheta_j) = \Prob\{T_{ij} > t\}, \quad
    f_j(t; \vtheta_j) = -R_j'(t; \vtheta_j), \quad
    h_j(t; \vtheta_j) = \frac{f_j(t; \vtheta_j)}{R_j(t; \vtheta_j)}.
\end{equation}
Independence yields the series system hazard
$h_T(t; \vtheta) = \sum_{j=1}^m h_j(t; \vtheta_j)$ and survival
$R_T(t; \vtheta) = \prod_{j=1}^m R_j(t; \vtheta_j)$.

Given that the system failed at time $t$, the conditional probability that
component $j$ caused the failure is
\begin{equation}
\label{eq:cond-failure}
    \Prob\{K_i = j \mid T_i = t\} = \frac{h_j(t; \vtheta_j)}{%
    \sum_{\ell=1}^m h_\ell(t; \vtheta_\ell)},
\end{equation}
which follows from the joint density $f_{T,K}(t,k) = h_k(t;\vtheta_k)\,
R_T(t;\vtheta)$ and Bayes' theorem~\citep{towell2023reliability}.

\subsection{Masked Data}
\label{sec:masked-data}

For each system $i$, we observe:
\begin{itemize}
    \item $S_i = \min\{T_i, \tau_i\}$: right-censored system lifetime,
    \item $\delta_i = \ind{T_i \leq \tau_i}$: event indicator
        ($1$ if failure observed),
    \item $C_i \subseteq \{1, \ldots, m\}$: candidate set
        (only meaningful when $\delta_i = 1$).
\end{itemize}
The true cause of failure $K_i$ and the individual component lifetimes
$(T_{i1}, \ldots, T_{im})$ are latent.

\subsection{Conditions C1, C2, C3}
\label{sec:conditions}

Tractable likelihood-based inference requires conditions on the relationship
between the latent cause $K_i$ and the observed candidate set
$C_i$~\citep{Usher-1988, Lin-1993, Huairu-2013}:

\begin{condition}[C1: Failed Component in Candidate Set]
\label{cond:C1}
$\Prob\{K_i \in C_i\} = 1$.
\end{condition}

\begin{condition}[C2: Non-Informative Masking]
\label{cond:C2}
For all $j, j' \in c$:
\begin{equation}
    \Prob\{C_i = c \mid T_i = t, K_i = j\} =
    \Prob\{C_i = c \mid T_i = t, K_i = j'\}.
\end{equation}
\end{condition}

\begin{condition}[C3: Parameter-Independent Masking]
\label{cond:C3}
$\Prob\{C_i = c \mid T_i = t, K_i = j\}$ does not depend on $\vtheta$.
\end{condition}

C1 is a minimal correctness requirement. C3 is often reasonable when the
diagnostic process is fixed independently of the reliability parameters. C2 is
the strongest assumption: it requires that the diagnostic process which
generates candidate sets reveals nothing about the failure cause beyond C1.
This paper studies the consequences of C2 failure.

\subsection{Likelihood Under C1-C2-C3}
\label{sec:like-C123}

\begin{theorem}[Likelihood Under C1-C2-C3 {\citep{towell2023reliability}}]
\label{thm:like-C123}
Under Conditions C1, C2, and C3, the likelihood contribution from observation
$i$ is
\begin{equation}
\label{eq:like-C123}
    L_i(\vtheta) \propto \prod_{\ell=1}^m R_\ell(s_i; \vtheta_\ell) \cdot
    \left[\sum_{k \in c_i} h_k(s_i; \vtheta_k)\right]^{\delta_i}.
\end{equation}
\end{theorem}

The masking probability factors out and cancels from the likelihood, leaving a
tractable expression for MLE. The complete log-likelihood for $n$ independent
systems is
\begin{equation}
\label{eq:loglike-C123}
    \ell(\vtheta) = \sum_{i=1}^n \left[
        \sum_{j=1}^m \log R_j(s_i; \vtheta_j) +
        \delta_i \log\left(\sum_{k \in c_i} h_k(s_i; \vtheta_k)\right)
    \right].
\end{equation}

\paragraph{Exponential specialization.}
When each component has exponential lifetime $T_{ij} \sim \text{Exp}(\theta_j)$
with rate $\theta_j > 0$, we have $h_j(t) = \theta_j$ and
$R_j(t) = e^{-\theta_j t}$, giving
\begin{equation}
\label{eq:loglike-exp}
    \ell(\vtheta) = \sum_{i=1}^n \left[
        -s_i \sum_{j=1}^m \theta_j +
        \delta_i \log\left(\sum_{k \in c_i} \theta_k\right)
    \right].
\end{equation}

\paragraph{Weibull specialization.}
When each component has Weibull lifetime with shape $k_j$ and scale
$\lambda_j$, the hazard is $h_j(t) = (k_j/\lambda_j)(t/\lambda_j)^{k_j-1}$
and $R_j(t) = \exp[-(t/\lambda_j)^{k_j}]$. The log-likelihood becomes
\begin{equation}
\label{eq:loglike-wei}
    \ell(\vtheta) = \sum_{i=1}^n \left[
        -\sum_{j=1}^m \left(\frac{s_i}{\lambda_j}\right)^{k_j} +
        \delta_i \log\left(\sum_{k \in c_i} h_k(s_i)\right)
    \right],
\end{equation}
where $\vtheta = (k_1, \lambda_1, \ldots, k_m, \lambda_m)$.

\subsection{Prior Work on Dependent Masking}
\label{sec:prior-work}

Several authors have proposed models that relax C2 by explicitly specifying a
masking mechanism that depends on the failure cause.

\citet{LinGuess1994} introduced proportional masking probabilities for
two-component systems, deriving closed-form MLEs under the assumption that the
masking probability for the true cause exceeds that for the non-failed
component by a fixed ratio. \citet{Guttman1995} extended two-component
dependent masking to a Bayesian framework with conjugate priors.
\citet{Mukhopadhyay2006} developed EM-based maximum likelihood estimation for
general $m$-component systems where masking probabilities depend on the cause
of failure, establishing conditions for identifiability of the joint
parameter. \citet{CraiuReiser2010} studied conditional masking probability
models and proved identifiability results under specific structural
assumptions on the masking mechanism.

These approaches share a common strategy: they replace C2 with a specific
parametric masking model and estimate its parameters jointly with the
component reliability parameters. The implicit assumption is that the masking
model is correctly specified. However, the masking mechanism is itself a
diagnostic process that is rarely known in detail. A natural complementary
question---one not addressed in the prior literature---is: \emph{how sensitive
is the standard C1-C2-C3 estimator to violations of C2 when no alternative
masking model is adopted?} This is the question we study.
