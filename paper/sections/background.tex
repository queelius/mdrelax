%% ============================================================================
\section{Background}
\label{sec:background}
%% ============================================================================

\subsection{Series System Model}
\label{sec:series-model}

Consider a series system composed of $m$ components. The lifetime of the
$i$-th system is
\begin{equation}
    T_i = \min\{T_{i1}, T_{i2}, \ldots, T_{im}\},
\end{equation}
where $T_{ij}$ denotes the lifetime of the $j$-th component in the $i$-th
system. Component lifetimes are assumed independent with parametric
distributions indexed by $\vtheta_j$; the full parameter vector is
$\vtheta = (\vtheta_1, \ldots, \vtheta_m) \in \vOmega$.

\begin{definition}[Component Distribution Functions]
\label{def:component-dist}
For component $j$ with parameter $\vtheta_j$:
\begin{align}
    R_j(t; \vtheta_j) &= \Prob\{T_{ij} > t\} && \text{(reliability function)}, \\
    f_j(t; \vtheta_j) &= -\frac{d}{dt}R_j(t; \vtheta_j) && \text{(density function)}, \\
    h_j(t; \vtheta_j) &= \frac{f_j(t; \vtheta_j)}{R_j(t; \vtheta_j)} && \text{(hazard function)}.
\end{align}
\end{definition}

For the series system, these functions combine as follows:

\begin{theorem}[Series System Distribution Functions]
\label{thm:series-dist}
The series system has:
\begin{align}
    R_{T_i}(t; \vtheta) &= \prod_{j=1}^m R_j(t; \vtheta_j), \\
    h_{T_i}(t; \vtheta) &= \sum_{j=1}^m h_j(t; \vtheta_j), \\
    f_{T_i}(t; \vtheta) &= h_{T_i}(t; \vtheta) \cdot R_{T_i}(t; \vtheta).
\end{align}
\end{theorem}

The proof follows from the independence of component lifetimes and standard
arguments \citep{towell2023reliability}.

\subsection{Component Cause of Failure}
\label{sec:cause-failure}

Let $K_i \in \{1, \ldots, m\}$ denote the index of the component that caused
system $i$ to fail. Since the system fails when the first component fails,
$K_i = \arg\min_j T_{ij}$.

\begin{theorem}[Joint Distribution of $(T_i, K_i)$]
\label{thm:joint-T-K}
The joint distribution of system lifetime and component cause of failure is:
\begin{equation}
\label{eq:joint-T-K}
    f_{T_i, K_i}(t, k; \vtheta) = h_k(t; \vtheta_k) \prod_{\ell=1}^m R_\ell(t; \vtheta_\ell).
\end{equation}
\end{theorem}

\begin{corollary}[Conditional Failure Probability]
\label{cor:cond-failure}
Given that the system failed at time $t$, the probability that component $j$
caused the failure is:
\begin{equation}
\label{eq:cond-failure}
    \Prob\{K_i = j \mid T_i = t\} = \frac{h_j(t; \vtheta_j)}{\sum_{\ell=1}^m h_\ell(t; \vtheta_\ell)}.
\end{equation}
\end{corollary}

This probability plays a central role in masked data analysis, as it
represents the ``true'' probability that each component failed, which the
masking mechanism partially obscures.

\subsection{Masked Data Structure}
\label{sec:masked-data}

\begin{definition}[Observed Data]
\label{def:observed-data}
For each system $i$, we observe:
\begin{itemize}
    \item $S_i = \min\{T_i, \tau_i\}$: Right-censored system lifetime,
    \item $\delta_i = \ind{T_i \leq \tau_i}$: Event indicator
        ($1$ if failure observed, $0$ if censored),
    \item $C_i \subseteq \{1, \ldots, m\}$: Candidate set
        (only observed when $\delta_i = 1$).
\end{itemize}
The latent (unobserved) variables are:
\begin{itemize}
    \item $K_i \in \{1, \ldots, m\}$: Index of failed component,
    \item $(T_{i1}, \ldots, T_{im})$: Component failure times.
\end{itemize}
\end{definition}

\subsection{Traditional Conditions C1, C2, C3}
\label{sec:conditions}

The existing literature \citep{Usher-1988,Lin-1993,Huairu-2013} establishes
tractable likelihood-based inference under three conditions:

\begin{condition}[C1: Failed Component in Candidate Set]
\label{cond:C1}
The candidate set always contains the failed component:
\begin{equation}
    \Prob\{K_i \in C_i\} = 1.
\end{equation}
\end{condition}

\begin{condition}[C2: Non-Informative Masking]
\label{cond:C2}
Given the failure time and that the failed component is in a candidate set $c$,
the probability of observing $c$ does not depend on which component in $c$
failed:
\begin{equation}
    \Prob\{C_i = c \mid T_i = t, K_i = j\} =
    \Prob\{C_i = c \mid T_i = t, K_i = j'\}
\end{equation}
for all $j, j' \in c$.
\end{condition}

\begin{condition}[C3: Parameter-Independent Masking]
\label{cond:C3}
The masking probabilities do not depend on the system parameters:
\begin{equation}
    \Prob\{C_i = c \mid T_i = t, K_i = j\} = \beta_i(c, t, j),
\end{equation}
where $\beta_i$ does not depend on $\vtheta$.
\end{condition}

\subsection{Likelihood Under C1, C2, C3}
\label{sec:like-C123}

Under all three conditions, the likelihood admits a tractable form:

\begin{theorem}[Likelihood Under C1-C2-C3]
\label{thm:like-C123}
Under Conditions C1, C2, and C3, the likelihood contribution from an
uncensored observation $(s_i, c_i)$ is proportional to:
\begin{equation}
\label{eq:like-C123}
    L_i(\vtheta) \propto \prod_{\ell=1}^m R_\ell(s_i; \vtheta_\ell) \cdot
    \sum_{k \in c_i} h_k(s_i; \vtheta_k).
\end{equation}
For a censored observation with lifetime $s_i$:
\begin{equation}
    L_i(\vtheta) = \prod_{\ell=1}^m R_\ell(s_i; \vtheta_\ell).
\end{equation}
\end{theorem}

\begin{proof}
By the chain rule:
\begin{equation}
    f_{T_i, C_i}(t, c; \vtheta) =
    \sum_{k=1}^m f_{T_i, K_i}(t, k; \vtheta) \cdot
    \Prob_{\vtheta}\{C_i = c \mid T_i = t, K_i = k\}.
\end{equation}
Under C1, terms with $k \notin c$ vanish. Under C2, the masking probability
is constant over $k \in c$, so:
\begin{equation}
    f_{T_i, C_i}(t, c; \vtheta) =
    \Prob\{C_i = c \mid T_i = t, K_i \in c\} \cdot
    \prod_{\ell=1}^m R_\ell(t; \vtheta_\ell) \sum_{k \in c} h_k(t; \vtheta_k).
\end{equation}
Under C3, the masking probability does not depend on $\vtheta$, yielding
the proportionality in \eqref{eq:like-C123}.
\end{proof}

The complete log-likelihood for $n$ independent systems is:
\begin{equation}
\label{eq:loglike-C123}
    \ell(\vtheta) = \sum_{i=1}^n \left[
        \sum_{j=1}^m \log R_j(s_i; \vtheta_j) +
        \delta_i \log\left(\sum_{k \in c_i} h_k(s_i; \vtheta_k)\right)
    \right].
\end{equation}

\subsection{Related Work}
\label{sec:related}

The masked data problem in series systems was introduced by
\citet{Usher-1988}, who developed MLE methods for exponential components.
\citet{Lin-1993} extended this to Weibull components with exact maximum
likelihood. \citet{Huairu-2013} (Guo et al.) provided simulation studies
validating the approach under various masking scenarios.

The informative censoring literature in survival analysis
\citep{klein2005survival,cox1972regression} addresses related issues where
the censoring mechanism depends on covariates or outcomes. However, the
candidate set structure in masked data creates a distinct problem not fully
addressed by standard informative censoring methods.

The competing risks framework \citep{Agustin-2011} provides another
perspective, viewing component failures as competing causes of system failure.
However, standard competing risks methods assume the cause is observed, whereas
masked data only provides partial information through candidate sets.

Our work extends the C1-C2-C3 framework by explicitly modeling departures from
C2 and C3, providing both theoretical analysis and practical estimation methods.
