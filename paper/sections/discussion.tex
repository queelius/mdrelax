%% ============================================================================
\section{Discussion}
\label{sec:discussion}
%% ============================================================================

\subsection{When Does C2 Matter?}
\label{sec:when-C2}

Our results delineate two distinct regimes for masked series system inference:

\paragraph{System-level inference: C2 rarely matters.}
If the goal is to estimate the system hazard rate, the mean time to failure,
or the system reliability function, the standard C1-C2-C3 model is robust to
C2 violations. The system hazard remained approximately unbiased across all
severity levels in our 5-component Weibull system. This robustness arises from
the structure of the misspecified score: summing over components eliminates the
masking weights, so the total hazard estimating equation is approximately
correct regardless of the masking mechanism (\Cref{thm:misspec-C2}).

\paragraph{Component-level inference: C2 matters.}
If the goal is to estimate individual component parameters---for example,
to identify the most unreliable component or to plan component-specific
maintenance---the standard model can produce substantially biased estimates
when C2 is violated. The bias grows with violation severity, and coverage
degrades to the point where confidence intervals are misleading. Scale
parameters are particularly affected, with the Weibull shape-scale interaction
amplifying misspecification bias nonlinearly.

\subsection{Practical Guidance}
\label{sec:guidance}

Since the masking mechanism is a diagnostic process that is rarely known in
detail, the practitioner typically cannot determine whether C2 holds. We
recommend the following approach:

\begin{enumerate}
    \item \textbf{Fit the standard model.} The C1-C2-C3 MLE is well-understood,
        computationally straightforward, and yields reliable system-level
        estimates regardless of C2 status.

    \item \textbf{Assess the inferential target.} If only system-level
        quantities are needed, the standard model suffices and no further
        action is required.

    \item \textbf{Run sensitivity analysis for component-level inference.}
        Construct a family of plausible $\mathbf{P}$ matrices representing
        different degrees and directions of masking asymmetry. Refit the
        model with known $\mathbf{P}$ under each scenario. If
        the component-level conclusions are stable across the family,
        C2 violations are not a concern.

    \item \textbf{Report sensitivity bounds.} When conclusions are sensitive to
        the assumed masking structure, report the range of estimates across the
        $\mathbf{P}$ family as sensitivity bounds, analogous to sensitivity
        analysis in causal inference.
\end{enumerate}

\subsection{Limitations}
\label{sec:limitations}

Several limitations should be acknowledged:

\begin{itemize}
    \item \textbf{Perturbation family.} Our simulation sweep uses the Bernoulli
        $\mathbf{P}(s)$ model as the sole perturbation device. Other violation
        structures---for example, time-dependent masking where $\mathbf{P}$
        varies with $t$---could behave differently.

    \item \textbf{Lifetime distributions.} We study Weibull components, which
        subsume exponential as a special case ($k = 1$). Other parametric
        families (e.g., log-normal, Gompertz) may exhibit different sensitivity
        profiles.

    \item \textbf{Direction dependence.} The specific direction matrix
        $\mathbf{D}$ determines which parameters are over- or under-estimated.
        Different $\mathbf{D}$ matrices would produce different bias patterns,
        though the system-hazard robustness should persist by
        \Cref{thm:misspec-C2}.

    \item \textbf{Sample size.} Our simulations use $n = 500$. The relative
        bias would be similar at larger sample sizes (misspecification bias
        does not shrink with $n$), but coverage degradation would be more
        pronounced as confidence intervals narrow.
\end{itemize}
