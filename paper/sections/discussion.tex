%% ============================================================================
\section{Discussion}
\label{sec:discussion}
%% ============================================================================

\subsection{When to Use Relaxed Models}
\label{sec:when-relaxed}

The theoretical and simulation results suggest the following practical
guidance for choosing between standard C1-C2-C3 models and relaxed alternatives.

\subsubsection{Use Standard C1-C2-C3 When:}

\begin{enumerate}
    \item \textbf{Masking mechanism is genuinely uninformative.} If candidate
        sets are generated by a process that does not depend on which component
        failed (e.g., random equipment availability for testing), C2 holds.

    \item \textbf{Masking probabilities are unknown.} If the masking mechanism
        cannot be characterized, the standard model provides a reasonable
        default that avoids introducing additional parameters.

    \item \textbf{Sample size is small.} Even if masking is slightly
        informative, the bias may be dominated by sampling variability for
        small $n$. The simpler model may provide more stable estimates.

    \item \textbf{Primary interest is in relative component reliability.}
        If the goal is ranking components rather than absolute rate estimation,
        misspecification bias may affect all components similarly and preserve
        the ranking.
\end{enumerate}

\subsubsection{Consider Relaxed Models When:}

\begin{enumerate}
    \item \textbf{Masking mechanism is known to be informative (C2).} If
        diagnostic procedures systematically favor certain components (e.g.,
        those that ``look bad'' at the failure time), C2 is violated and
        bias will result.

    \item \textbf{Masking depends on component parameters (C3).} If
        inclusion probabilities vary with the reliability parameters
        themselves (e.g., weaker components are more likely to appear in
        candidate sets), C3 is violated. The simulation studies
        (\Cref{sec:sim-c3}) show that the resulting bias grows with the
        degree of parameter-dependence, though the simpler C1-C2-C3 model
        may still have competitive RMSE due to a bias-variance tradeoff.

    \item \textbf{Masking probabilities can be estimated.} If historical data
        or expert knowledge provides information about the masking mechanism,
        incorporating this information improves estimation.

    \item \textbf{Sample size is large enough to support additional parameters.}
        Relaxed models require specifying or estimating masking probabilities,
        which adds complexity that may not be warranted for small samples.

    \item \textbf{Identifiability concerns are present.} As shown in
        \Cref{thm:ident-inform}, informative masking can resolve
        identifiability issues that arise under standard conditions.
\end{enumerate}

\subsection{Practical Guidance}
\label{sec:guidance}

Based on our analysis, we recommend the following workflow:

\begin{enumerate}
    \item \textbf{Assess the masking mechanism.} Before estimation, consider
        how candidate sets are generated. Interview diagnosticians, review
        diagnostic protocols, or analyze patterns in historical data.

    \item \textbf{Check for block structure.} Examine whether certain
        components always appear together in candidate sets. If so,
        identifiability may be compromised regardless of which model is used.

    \item \textbf{Perform sensitivity analysis.} Fit models under both
        C1-C2-C3 and plausible relaxed assumptions. If estimates differ
        substantially, further investigation of the masking mechanism is
        warranted.

    \item \textbf{Use simulation to assess impact.} Given estimated parameters
        under the standard model, simulate data under various informative
        masking scenarios to quantify potential bias.

    \item \textbf{Report uncertainty appropriately.} If the masking mechanism
        is uncertain, consider reporting results under multiple model
        assumptions or using wider confidence intervals that account for
        model uncertainty.
\end{enumerate}

\subsection{Limitations}
\label{sec:limitations}

Our analysis has several limitations:

\begin{enumerate}
    \item \textbf{Limited distribution scope.} While we extend the analysis
        to Weibull components (\Cref{sec:sim-weibull}), the closed-form
        Fisher information results focus on exponential systems. Other
        lifetime distributions (log-normal, gamma) may exhibit different
        misspecification patterns due to differing hazard structures.

    \item \textbf{Known masking probabilities.} Our relaxed models assume
        masking probabilities are known. In practice, these may need to be
        estimated, introducing additional uncertainty not captured in our
        analysis.

    \item \textbf{Independence assumption.} We assume masking for different
        observations is independent. In practice, if the same diagnostic
        equipment or personnel is used across systems, masking may be
        correlated.

    \item \textbf{Parametric masking models.} Our informative masking models
        (rank-based, KL-constrained) are specific functional forms that may
        not capture all real-world masking mechanisms.

    \item \textbf{Simulation scope.} The simulation studies cover a limited
        range of configurations. Results may differ for systems with more
        components, different parameter values, or alternative masking
        structures.
\end{enumerate}

\subsection{Future Directions}
\label{sec:future}

Several extensions would strengthen this work:

\begin{enumerate}
    \item \textbf{Semiparametric methods.} Develop estimation approaches that
        avoid fully specifying the masking mechanism, perhaps using
        nonparametric or empirical likelihood methods.

    \item \textbf{Model selection.} Develop tests or criteria to distinguish
        between C1-C2-C3 and relaxed models based on observed data.

    \item \textbf{Bayesian extensions.} Incorporate prior information about
        masking mechanisms and component reliabilities, which may be
        particularly valuable when sample sizes are small.

    \item \textbf{Sequential estimation.} For systems observed over time,
        develop methods that update masking probability estimates as data
        accumulates.

    \item \textbf{Additional lifetime distributions.} While our Weibull
        extension (\Cref{sec:sim-weibull}) validates the framework beyond
        exponential components, distributions with non-monotone hazards
        (e.g., log-normal) may present distinct challenges.

    \item \textbf{R package documentation.} Expand the
        \texttt{mdrelax} package with
        vignettes demonstrating practical application of these methods.
\end{enumerate}
