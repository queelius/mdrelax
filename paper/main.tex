%% Sensitivity of Series System Reliability Estimation
%% to the Non-Informative Masking Assumption
%% Main document

\documentclass[11pt,letterpaper]{article}

%% ============================================================================
%% Packages
%% ============================================================================

% Mathematics
\usepackage{amsmath,amsthm,amssymb}
\usepackage{bm}
\usepackage{mathtools}

% Graphics and figures
\usepackage{graphicx}
\graphicspath{{../inst/simulations/figures/}}
\usepackage{float}
\usepackage{subcaption}

% Tables
\usepackage{booktabs}
\usepackage{multirow}

% Typography and layout
\usepackage[margin=1in]{geometry}
\usepackage{setspace}
\usepackage{microtype}

% References and links
\usepackage[numbers,sort&compress]{natbib}
\usepackage{hyperref}
\usepackage{cleveref}

% Algorithms (if needed)
\usepackage{algorithm}
\usepackage{algpseudocode}

% Colors for comments (remove in final version)
\usepackage{xcolor}
\newcommand{\todo}[1]{\textcolor{red}{[TODO: #1]}}
\newcommand{\placeholder}[1]{\textcolor{blue}{[PLACEHOLDER: #1]}}

%% ============================================================================
%% Theorem Environments
%% ============================================================================

\newtheorem{theorem}{Theorem}[section]
\newtheorem{lemma}[theorem]{Lemma}
\newtheorem{proposition}[theorem]{Proposition}
\newtheorem{corollary}[theorem]{Corollary}
\newtheorem{definition}[theorem]{Definition}
\newtheorem{condition}{Condition}
\newtheorem{assumption}{Assumption}

\theoremstyle{remark}
\newtheorem{remark}{Remark}[section]
\newtheorem{example}{Example}[section]

%% ============================================================================
%% Notation Macros
%% ============================================================================

% Vectors and matrices
\newcommand{\vtheta}{\bm{\theta}}
\newcommand{\vlambda}{\bm{\lambda}}
\newcommand{\vOmega}{\bm{\Omega}}
\newcommand{\vbeta}{\bm{\beta}}

% Calligraphic letters
\newcommand{\calC}{\mathcal{C}}
\newcommand{\calD}{\mathcal{D}}
\newcommand{\calL}{\mathcal{L}}
\newcommand{\calI}{\mathcal{I}}

% Operators
\newcommand{\Prob}{\mathrm{Pr}}
\newcommand{\E}{\mathbb{E}}
\newcommand{\Var}{\mathrm{Var}}
\newcommand{\Cov}{\mathrm{Cov}}
\newcommand{\KL}{\mathrm{KL}}

% Fisher information
\newcommand{\FIM}{\mathcal{I}}

% Indicator function
\newcommand{\ind}[1]{\mathbf{1}_{#1}}

%% ============================================================================
%% Document Information
%% ============================================================================

\title{Sensitivity of Series System Reliability Estimation\\
to the Non-Informative Masking Assumption}

\author{
    Alex Towell\\
    \texttt{lex@metafunctor.com}
}

\date{\today}

%% ============================================================================
%% Document
%% ============================================================================

\begin{document}

\maketitle

\begin{abstract}
Maximum likelihood estimation for series systems with masked failure data
relies on the non-informative masking assumption (Condition~C2): that the
candidate set reported for a failed component is independent of the true cause
of failure. This assumption is rarely verifiable and frequently suspect. We
characterize the bias that arises when C2 is violated but assumed to hold,
proving that the MLE converges to a pseudo-true parameter whose individual
component parameters are biased while the system hazard remains consistent.
Using a Bernoulli masking model that generates controlled C2 violations of
varying severity, we conduct a systematic simulation sweep on a 5-component
Weibull series system. The results establish a practical breakdown boundary:
system-level reliability estimates are robust to C2 violations across the full
severity range, but individual component estimates degrade with severity,
with scale parameters showing order-of-magnitude bias under strong
informative masking. Since the masking mechanism and component parameters are
jointly non-identifiable, we recommend sensitivity analysis over model
elaboration. Software is available in the \texttt{mdrelax} R package.
\end{abstract}

\noindent\textbf{Keywords:}
Series systems, masked failure data, non-informative masking, model
misspecification, sensitivity analysis, maximum likelihood estimation

%% ============================================================================
%% Main Sections
%% ============================================================================

%% ============================================================================
\section{Introduction}
\label{sec:introduction}
%% ============================================================================

Estimating the reliability of individual components in a series system presents
a fundamental challenge in reliability engineering: system-level failure data
is observable, but component-level failure causes are often masked. When a
series system fails, diagnostic procedures may identify only a \emph{candidate
set} of components that could have caused the failure, rather than pinpointing
the exact failed component \citep{Usher-1988}. This masking, combined with
right-censoring of system lifetimes, complicates statistical inference about
component reliability parameters.

Prior work on masked data in series systems has established a tractable
likelihood framework under three conditions \citep{towell2023reliability}:
\begin{itemize}
    \item \textbf{C1:} The failed component is always contained in the
        candidate set.
    \item \textbf{C2:} Masking is non-informative---conditional on the failure
        time and candidate set, each component in the candidate set is equally
        likely to have failed.
    \item \textbf{C3:} The masking mechanism does not depend on the system
        parameters $\vtheta$.
\end{itemize}
Under these conditions, the likelihood has a simple closed form that enables
efficient maximum likelihood estimation (MLE). However, practical diagnostic
systems may violate C2 and C3:
\begin{itemize}
    \item Experienced technicians may preferentially include components that
        ``seem likely'' to have failed based on failure time characteristics,
        violating C2.
    \item Diagnostic algorithms based on reliability rankings may systematically
        favor certain components, with the ranking depending on the true
        parameters, violating C3.
\end{itemize}

This paper develops the theoretical and computational framework for
likelihood-based inference when C2 and/or C3 are relaxed while maintaining C1.
Our contributions are:
\begin{enumerate}
    \item \textbf{General likelihood framework.} We derive the likelihood under
        C1 alone, showing how informative and parameter-dependent masking
        modify the standard likelihood structure (\Cref{sec:relaxed}).

    \item \textbf{Practical masking models.} We introduce the rank-based
        informative masking model and the Bernoulli candidate set model with
        KL-divergence constraints, which provide interpretable parameterizations
        of non-standard masking (\Cref{sec:relaxed}).

    \item \textbf{Identifiability analysis.} We establish conditions under which
        parameters remain identifiable when standard conditions fail, including
        the surprising result that informative masking can \emph{improve}
        identifiability in certain cases (\Cref{sec:identifiability}).

    \item \textbf{Efficiency comparison.} We derive the Fisher information
        matrix under relaxed conditions for exponential series systems,
        enabling precise comparison of estimation efficiency
        (\Cref{sec:identifiability}).

    \item \textbf{Simulation studies.} We quantify the bias from incorrectly
        assuming C2 when masking is informative, and demonstrate the improved
        estimation achievable when the masking structure is properly modeled
        (\Cref{sec:simulations}).
\end{enumerate}

The remainder of this paper is organized as follows. \Cref{sec:background}
reviews series systems, masked data, and the standard C1-C2-C3 likelihood.
\Cref{sec:relaxed} develops the likelihood under relaxed conditions.
\Cref{sec:identifiability} analyzes identifiability and Fisher information.
\Cref{sec:simulations} presents simulation studies. \Cref{sec:discussion}
discusses practical implications, and \Cref{sec:conclusion} concludes.

%% ============================================================================
\section{Background}
\label{sec:background}
%% ============================================================================

\subsection{Series System Model}
\label{sec:series-model}

Consider a series system composed of $m$ components. The lifetime of the
$i$-th system is
\begin{equation}
    T_i = \min\{T_{i1}, T_{i2}, \ldots, T_{im}\},
\end{equation}
where $T_{ij}$ denotes the lifetime of the $j$-th component in the $i$-th
system. Component lifetimes are assumed independent with parametric
distributions indexed by $\vtheta_j$; the full parameter vector is
$\vtheta = (\vtheta_1, \ldots, \vtheta_m) \in \vOmega$.

\begin{definition}[Component Distribution Functions]
\label{def:component-dist}
For component $j$ with parameter $\vtheta_j$:
\begin{align}
    R_j(t; \vtheta_j) &= \Prob\{T_{ij} > t\} && \text{(reliability function)}, \\
    f_j(t; \vtheta_j) &= -\frac{d}{dt}R_j(t; \vtheta_j) && \text{(density function)}, \\
    h_j(t; \vtheta_j) &= \frac{f_j(t; \vtheta_j)}{R_j(t; \vtheta_j)} && \text{(hazard function)}.
\end{align}
\end{definition}

For the series system, these functions combine as follows:

\begin{theorem}[Series System Distribution Functions]
\label{thm:series-dist}
The series system has:
\begin{align}
    R_{T_i}(t; \vtheta) &= \prod_{j=1}^m R_j(t; \vtheta_j), \\
    h_{T_i}(t; \vtheta) &= \sum_{j=1}^m h_j(t; \vtheta_j), \\
    f_{T_i}(t; \vtheta) &= h_{T_i}(t; \vtheta) \cdot R_{T_i}(t; \vtheta).
\end{align}
\end{theorem}

The proof follows from the independence of component lifetimes and standard
arguments \citep{towell2023reliability}.

\subsection{Component Cause of Failure}
\label{sec:cause-failure}

Let $K_i \in \{1, \ldots, m\}$ denote the index of the component that caused
system $i$ to fail. Since the system fails when the first component fails,
$K_i = \arg\min_j T_{ij}$.

\begin{theorem}[Joint Distribution of $(T_i, K_i)$]
\label{thm:joint-T-K}
The joint distribution of system lifetime and component cause of failure is:
\begin{equation}
\label{eq:joint-T-K}
    f_{T_i, K_i}(t, k; \vtheta) = h_k(t; \vtheta_k) \prod_{\ell=1}^m R_\ell(t; \vtheta_\ell).
\end{equation}
\end{theorem}

\begin{corollary}[Conditional Failure Probability]
\label{cor:cond-failure}
Given that the system failed at time $t$, the probability that component $j$
caused the failure is:
\begin{equation}
\label{eq:cond-failure}
    \Prob\{K_i = j \mid T_i = t\} = \frac{h_j(t; \vtheta_j)}{\sum_{\ell=1}^m h_\ell(t; \vtheta_\ell)}.
\end{equation}
\end{corollary}

This probability plays a central role in masked data analysis, as it
represents the ``true'' probability that each component failed, which the
masking mechanism partially obscures.

\subsection{Masked Data Structure}
\label{sec:masked-data}

\begin{definition}[Observed Data]
\label{def:observed-data}
For each system $i$, we observe:
\begin{itemize}
    \item $S_i = \min\{T_i, \tau_i\}$: Right-censored system lifetime,
    \item $\delta_i = \ind{T_i \leq \tau_i}$: Event indicator
        ($1$ if failure observed, $0$ if censored),
    \item $C_i \subseteq \{1, \ldots, m\}$: Candidate set
        (only observed when $\delta_i = 1$).
\end{itemize}
The latent (unobserved) variables are:
\begin{itemize}
    \item $K_i \in \{1, \ldots, m\}$: Index of failed component,
    \item $(T_{i1}, \ldots, T_{im})$: Component failure times.
\end{itemize}
\end{definition}

\subsection{Traditional Conditions C1, C2, C3}
\label{sec:conditions}

The existing literature \citep{Usher-1988,Lin-1993,Huairu-2013} establishes
tractable likelihood-based inference under three conditions:

\begin{condition}[C1: Failed Component in Candidate Set]
\label{cond:C1}
The candidate set always contains the failed component:
\begin{equation}
    \Prob\{K_i \in C_i\} = 1.
\end{equation}
\end{condition}

\begin{condition}[C2: Non-Informative Masking]
\label{cond:C2}
Given the failure time and that the failed component is in a candidate set $c$,
the probability of observing $c$ does not depend on which component in $c$
failed:
\begin{equation}
    \Prob\{C_i = c \mid T_i = t, K_i = j\} =
    \Prob\{C_i = c \mid T_i = t, K_i = j'\}
\end{equation}
for all $j, j' \in c$.
\end{condition}

\begin{condition}[C3: Parameter-Independent Masking]
\label{cond:C3}
The masking probabilities do not depend on the system parameters:
\begin{equation}
    \Prob\{C_i = c \mid T_i = t, K_i = j\} = \beta_i(c, t, j),
\end{equation}
where $\beta_i$ does not depend on $\vtheta$.
\end{condition}

\subsection{Likelihood Under C1, C2, C3}
\label{sec:like-C123}

Under all three conditions, the likelihood admits a tractable form:

\begin{theorem}[Likelihood Under C1-C2-C3]
\label{thm:like-C123}
Under Conditions C1, C2, and C3, the likelihood contribution from an
uncensored observation $(s_i, c_i)$ is proportional to:
\begin{equation}
\label{eq:like-C123}
    L_i(\vtheta) \propto \prod_{\ell=1}^m R_\ell(s_i; \vtheta_\ell) \cdot
    \sum_{k \in c_i} h_k(s_i; \vtheta_k).
\end{equation}
For a censored observation with lifetime $s_i$:
\begin{equation}
    L_i(\vtheta) = \prod_{\ell=1}^m R_\ell(s_i; \vtheta_\ell).
\end{equation}
\end{theorem}

\begin{proof}
By the chain rule:
\begin{equation}
    f_{T_i, C_i}(t, c; \vtheta) =
    \sum_{k=1}^m f_{T_i, K_i}(t, k; \vtheta) \cdot
    \Prob_{\vtheta}\{C_i = c \mid T_i = t, K_i = k\}.
\end{equation}
Under C1, terms with $k \notin c$ vanish. Under C2, the masking probability
is constant over $k \in c$, so:
\begin{equation}
    f_{T_i, C_i}(t, c; \vtheta) =
    \Prob\{C_i = c \mid T_i = t, K_i \in c\} \cdot
    \prod_{\ell=1}^m R_\ell(t; \vtheta_\ell) \sum_{k \in c} h_k(t; \vtheta_k).
\end{equation}
Under C3, the masking probability does not depend on $\vtheta$, yielding
the proportionality in \eqref{eq:like-C123}.
\end{proof}

The complete log-likelihood for $n$ independent systems is:
\begin{equation}
\label{eq:loglike-C123}
    \ell(\vtheta) = \sum_{i=1}^n \left[
        \sum_{j=1}^m \log R_j(s_i; \vtheta_j) +
        \delta_i \log\left(\sum_{k \in c_i} h_k(s_i; \vtheta_k)\right)
    \right].
\end{equation}

\subsection{Related Work}
\label{sec:related}

The masked data problem in series systems was introduced by
\citet{Usher-1988}, who developed MLE methods for exponential components.
\citet{Lin-1993} extended this to Weibull components with exact maximum
likelihood. \citet{Huairu-2013} (Guo et al.) provided simulation studies
validating the approach under various masking scenarios.

The informative censoring literature in survival analysis
\citep{klein2005survival,cox1972regression} addresses related issues where
the censoring mechanism depends on covariates or outcomes. However, the
candidate set structure in masked data creates a distinct problem not fully
addressed by standard informative censoring methods.

The competing risks framework \citep{Agustin-2011} provides another
perspective, viewing component failures as competing causes of system failure.
However, standard competing risks methods assume the cause is observed, whereas
masked data only provides partial information through candidate sets.

Our work extends the C1-C2-C3 framework by explicitly modeling departures from
C2 and C3, providing both theoretical analysis and practical estimation methods.

%% ============================================================================
\section{Sensitivity Framework}
\label{sec:framework}
%% ============================================================================

We now develop the tools needed to study the sensitivity of C1-C2-C3 inference
to C2 violations. We proceed in four steps: the general likelihood under C1
alone (\Cref{sec:like-C1}), a Bernoulli model for generating controlled C2
violations (\Cref{sec:bernoulli}), a misspecification theorem characterizing
the resulting bias (\Cref{sec:misspec}), and a non-identifiability result that
motivates sensitivity analysis over model estimation (\Cref{sec:identifiability}).

\subsection{Likelihood Under C1 Alone}
\label{sec:like-C1}

When only C1 holds, the masking probability enters the likelihood and cannot
be eliminated.

\begin{theorem}[Likelihood Under C1 Alone]
\label{thm:like-C1}
Under Condition C1 alone, the likelihood contribution from an uncensored
observation $(s_i, c_i)$ is:
\begin{equation}
\label{eq:like-C1}
    L_i(\vtheta) = \prod_{\ell=1}^m R_\ell(s_i; \vtheta_\ell) \cdot
    \sum_{k \in c_i} h_k(s_i; \vtheta_k) \cdot
    \Prob\{C_i = c_i \mid T_i = s_i, K_i = k\}.
\end{equation}
\end{theorem}

\begin{proof}
Under C1, $\Prob\{C_i = c \mid T_i = t, K_i = k\} = 0$ when $k \notin c$.
Summing over $K_i$:
\begin{align}
    f_{T_i, C_i}(t, c; \vtheta)
    &= \sum_{k=1}^m h_k(t; \vtheta_k) \prod_{\ell=1}^m R_\ell(t; \vtheta_\ell)
        \cdot \Prob\{C_i = c \mid T_i = t, K_i = k\} \notag \\
    &= \prod_{\ell=1}^m R_\ell(t; \vtheta_\ell) \cdot
        \sum_{k \in c} h_k(t; \vtheta_k)\,
        \Prob\{C_i = c \mid T_i = t, K_i = k\}. \qedhere
\end{align}
\end{proof}

Under C2, the masking probability $\Prob\{C_i = c \mid T_i = t, K_i = k\}$ is
constant over $k \in c$ and factors out of the sum. Under C3, it does not
depend on $\vtheta$ and can be dropped from the likelihood. When either
condition fails, the masking probability remains inside the sum, coupling the
inference about $\vtheta$ to the unknown masking mechanism.

\subsection{The Bernoulli Perturbation Model}
\label{sec:bernoulli}

To generate data with controlled C2 violations, we use a Bernoulli candidate
set model. We do not claim this model describes how masking works in practice.
Rather, it provides a device for producing data whose departure from C2 is
known and calibrated.

\begin{definition}[Bernoulli Masking Model]
\label{def:bernoulli}
Let $p_j(k) = \Prob\{j \in C_i \mid K_i = k\}$ denote the probability that
component $j$ is included in the candidate set given that component $k$
failed. Under C1, $p_j(j) = 1$ for all $j$. These probabilities form an
$m \times m$ matrix:
\begin{equation}
\label{eq:P-matrix}
    \mathbf{P} = \begin{pmatrix}
        1 & p_1(2) & \cdots & p_1(m) \\
        p_2(1) & 1 & \cdots & p_2(m) \\
        \vdots & \vdots & \ddots & \vdots \\
        p_m(1) & p_m(2) & \cdots & 1
    \end{pmatrix},
\end{equation}
where each component is included independently. The probability of observing
candidate set $c$ given $K_i = k$ is
\begin{equation}
\label{eq:pi-k-c}
    \pi_k(c) = \ind{k \in c} \prod_{j \in c \setminus \{k\}} p_j(k)
    \prod_{j \notin c} (1 - p_j(k)).
\end{equation}
\end{definition}

\begin{remark}[C2 in Terms of $\mathbf{P}$]
\label{rem:C2-P}
Condition C2 holds if and only if each row of $\mathbf{P}$ has constant
off-diagonal entries: $p_j(k) = p_j$ for all $k \neq j$. When the columns of
$\mathbf{P}$ differ, the masking mechanism ``knows'' which component failed,
violating C2.
\end{remark}

The likelihood under this model follows immediately from
\Cref{thm:like-C1}:

\begin{corollary}[Likelihood Under the Bernoulli Model]
\label{cor:bernoulli-like}
Under the Bernoulli masking model with known $\mathbf{P}$, the likelihood
contribution from an uncensored observation $(s_i, c_i)$ is:
\begin{equation}
\label{eq:bernoulli-like}
    L_i(\vtheta) = \prod_{\ell=1}^m R_\ell(s_i; \vtheta_\ell) \cdot
    \sum_{k \in c_i} h_k(s_i; \vtheta_k) \cdot \pi_k(c_i).
\end{equation}
\end{corollary}

\paragraph{Parameterizing violation severity.}
We generate a family of $\mathbf{P}$ matrices indexed by a scalar severity
parameter $s \in [0,1]$:
\begin{equation}
\label{eq:P-sweep}
    \mathbf{P}(s) = \mathbf{P}_0 + s\,\mathbf{D},
\end{equation}
where $\mathbf{P}_0$ is the uniform (C2-satisfying) matrix with all
off-diagonal entries equal to a base probability $p_0$, and $\mathbf{D}$ is a
fixed ``direction'' matrix with zero diagonal, whose off-diagonal entries
create column-wise asymmetry. Off-diagonal entries of $\mathbf{P}(s)$ are
clamped to $[0.05, 0.95]$ to maintain valid probabilities. At $s = 0$, C2
holds exactly; as $s$ increases, the columns of $\mathbf{P}$ diverge and the
masking becomes increasingly informative.

\subsection{Misspecification Under C2 Violation}
\label{sec:misspec}

We now characterize what happens when the standard C1-C2-C3 model is fitted
to data generated under C2 violation.

\begin{theorem}[Bias from C2 Misspecification]
\label{thm:misspec-C2}
Suppose data is generated under C1 and C3 with informative masking weights
$\pi_{k,c}$, but estimation is performed under C1-C2-C3 (assuming
non-informative masking). The MLE under the misspecified model converges in
probability to a pseudo-true parameter $\vtheta^{\dagger}$ satisfying
\begin{equation}
\label{eq:pseudo-true}
    \E_{\vtheta^*}\left[\frac{\partial \ell^{\mathrm{wrong}}}{\partial
    \theta_j}\bigg|_{\vtheta = \vtheta^{\dagger}}\right] = 0,
\end{equation}
where the expectation is taken under the true data-generating process with
parameter $\vtheta^*$. The pseudo-true parameter differs from $\vtheta^*$
unless C2 holds.

For exponential components with rates $\vtheta = (\theta_1, \ldots, \theta_m)$:
\begin{enumerate}
    \item[(a)] The total system hazard is approximately preserved:
        $\sum_j \theta_j^{\dagger} \approx \sum_j \theta_j^*$.
    \item[(b)] Individual rates absorb the masking asymmetry:
        components that are over-represented in candidate sets
        (due to asymmetric $\mathbf{P}$) have inflated $\theta_j^{\dagger}$,
        while under-represented components have deflated estimates.
\end{enumerate}
\end{theorem}

\begin{proof}
The misspecified score for exponential component $j$ is
\begin{equation}
    \frac{\partial \ell_i^{\mathrm{wrong}}}{\partial \theta_j} = -s_i +
    \delta_i\frac{\ind{j \in c_i}}{\sum_{k \in c_i} \theta_k}.
\end{equation}
The true score under C1-C3 (with masking weights $\pi_{k,c}$) is
\begin{equation}
    \frac{\partial \ell_i^{\mathrm{true}}}{\partial \theta_j} = -s_i +
    \delta_i\frac{\pi_{j,c_i}\,\ind{j \in c_i}}
         {\sum_{k \in c_i} \theta_k\,\pi_{k,c_i}}.
\end{equation}
At $\vtheta^*$, the true score has expectation zero. The misspecified score
replaces $\pi_{k,c}$ with uniform weights $1/|c|$. Setting
$\E[\partial\ell^{\mathrm{wrong}}/\partial\theta_j] = 0$ at
$\vtheta^{\dagger}$ yields a system whose solution differs from $\vtheta^*$
whenever the masking weights $\pi_{k,c}$ vary with $k$.

For part~(a), summing the misspecified score over all $j$ yields the total
hazard score
\begin{equation}
    \sum_j \frac{\partial \ell_i^{\mathrm{wrong}}}{\partial \theta_j}
    = -ms_i + \delta_i\frac{|c_i|}{\sum_{k \in c_i} \theta_k}.
\end{equation}
The analogous sum for the true score is
\begin{equation}
    \sum_j \frac{\partial \ell_i^{\mathrm{true}}}{\partial \theta_j}
    = -ms_i + \delta_i\frac{\sum_{k \in c_i} \pi_{k,c_i}}
    {\sum_{k \in c_i} \theta_k\,\pi_{k,c_i}}.
\end{equation}
When the masking weights satisfy
$\sum_{k \in c} \pi_{k,c} / \sum_{k \in c} \theta_k\pi_{k,c} \approx
|c| / \sum_{k \in c} \theta_k$, the total hazard score equations agree and
$\sum_j \theta_j^{\dagger} \approx \sum_j \theta_j^*$. This holds exactly
when C2 is satisfied and approximately when the masking asymmetry is
moderate.
\end{proof}

\begin{remark}[Interpretation]
The misspecified estimator acts as if each candidate in $c_i$ contributes
equally to the observed failure, regardless of the actual masking weights. When
a component is over-included in candidate sets (its column of $\mathbf{P}$ has
higher off-diagonal values), it ``claims credit'' for more failures than it
actually caused, inflating its estimated rate.
\end{remark}

\subsection{The Identifiability Trap}
\label{sec:identifiability}

A natural response to C2 uncertainty would be to estimate $\mathbf{P}$ jointly
with $\vtheta$ from the data. This fails.

\begin{theorem}[Non-Identifiability of $(\vtheta, \mathbf{P})$]
\label{thm:non-ident}
For exponential series systems with masked data, the joint parameter
$(\vtheta, \mathbf{P})$ is not identifiable from the observed data
$(s_i, c_i, \delta_i)_{i=1}^n$. Different combinations of $(\vtheta,
\mathbf{P})$ can yield equivalent likelihoods.

However, the total system hazard $\sum_{j=1}^m \theta_j$ remains identifiable
from the marginal system lifetime data.
\end{theorem}

This result is supported by both theoretical argument and simulation evidence.
Even with $n = 2000$ observations, the joint MLE of $(\vtheta, \mathbf{P})$
exhibits persistent bias with individual component rates confounded with the
off-diagonal entries of $\mathbf{P}$, while the total hazard
$\sum_j \hat\theta_j$ converges to the true value
(see~\Cref{sec:appendix-ident}).

\begin{remark}[Implications]
Since $\vtheta$ and $\mathbf{P}$ cannot be disentangled from the data, we
cannot ``just estimate'' the masking mechanism. This makes sensitivity
analysis the appropriate tool: fit the standard C1-C2-C3 model, then assess
how conclusions change under plausible perturbations of the masking structure
via different $\mathbf{P}$ matrices. If the substantive conclusions are stable,
C2 violations are not a concern for the application at hand.
\end{remark}

%% ============================================================================
\section{Simulation Studies}
\label{sec:simulations}
%% ============================================================================

We present simulation studies to (1) validate MLE performance under the
C1-C2-C3 Bernoulli masking model, (2) quantify the bias from incorrectly
assuming C2 when masking is informative, and (3) investigate identifiability
issues arising from correlated candidate sets.

\subsection{Experimental Design}
\label{sec:sim-design}

\subsubsection{System Configuration}

We consider exponential series systems with $m = 3$ components and true rate
parameters:
\begin{equation}
    \vlambda^* = (\lambda_1^*, \lambda_2^*, \lambda_3^*) = (1.0, 1.5, 2.0).
\end{equation}
These values represent a system where component 3 has the highest failure rate
(and thus contributes most to system failures), while component 1 is most
reliable.

\subsubsection{Data Generation}

For each simulation replicate:
\begin{enumerate}
    \item Generate component failure times $T_{ij} \sim \text{Exp}(\lambda_j^*)$
        for $i = 1, \ldots, n$ and $j = 1, \ldots, m$.
    \item Compute system failure times $T_i = \min_j T_{ij}$ and identify
        failed components $K_i = \arg\min_j T_{ij}$.
    \item Apply right-censoring at time $\tau$ (chosen to achieve approximately
        20\% censoring) to obtain observed lifetimes $S_i = \min(T_i, \tau)$
        and indicators $\delta_i$.
    \item Generate candidate sets using the specified masking model.
\end{enumerate}

\subsubsection{Masking Models}

We examine three masking scenarios:
\begin{enumerate}
    \item \textbf{C1-C2-C3 (Baseline):} Bernoulli model with $p = 0.3$ for
        all non-failed components.
    \item \textbf{Informative masking (Rank-based):} Masking probabilities
        depend on component failure time ranks, parameterized by
        informativeness parameter $\alpha \in \{0, 1, 2, 5, 10\}$.
    \item \textbf{Correlated candidate sets:} Candidate set indicators have
        correlation $\rho \in \{0, 0.1, 0.3, 0.5, 0.6, 0.8, 0.9\}$.
\end{enumerate}

\subsubsection{Performance Metrics}

We evaluate:
\begin{itemize}
    \item \textbf{Bias:} $\text{Bias}(\hat{\lambda}_j) =
        \E[\hat{\lambda}_j] - \lambda_j^*$
    \item \textbf{Root mean squared error (RMSE):}
        $\text{RMSE}(\hat{\lambda}_j) = \sqrt{\E[(\hat{\lambda}_j - \lambda_j^*)^2]}$
    \item \textbf{Coverage probability:} Proportion of 95\% confidence
        intervals containing $\lambda_j^*$
    \item \textbf{RMSE ratio:} $\text{RMSE}_{\text{misspec}} /
        \text{RMSE}_{\text{correct}}$
\end{itemize}

\subsection{Study 1: MLE Performance Under Bernoulli Masking}
\label{sec:sim-kl}

We first validate MLE performance under the correctly specified C1-C2-C3
Bernoulli masking model across sample sizes $n \in \{50, 100, 200\}$ with
$B = 200$ Monte Carlo replicates.

\subsubsection{Results}

Table~\ref{tab:mle_performance} presents the estimation results.

\begin{table}[htbp]
\centering
\caption{Maximum Likelihood Estimation Performance by Sample Size}
\label{tab:mle_performance}
\begin{tabular}{cccccc}
\toprule
$n$ & Parameter & Bias & RMSE & Coverage & Mean CI Width \\
\midrule
50 & $\lambda_1$ & 0.017 & 0.477 & 0.920 & 1.727 \\
 & $\lambda_2$ & 0.007 & 0.511 & 0.935 & 1.952 \\
 & $\lambda_3$ & 0.085 & 0.557 & 0.945 & 2.197 \\
\midrule
100 & $\lambda_1$ & 0.016 & 0.318 & 0.935 & 1.175 \\
 & $\lambda_2$ & 0.055 & 0.390 & 0.935 & 1.385 \\
 & $\lambda_3$ & $-0.037$ & 0.366 & 0.950 & 1.519 \\
\midrule
200 & $\lambda_1$ & $-0.005$ & 0.201 & 0.935 & 0.825 \\
 & $\lambda_2$ & 0.008 & 0.262 & 0.965 & 0.965 \\
 & $\lambda_3$ & $-0.033$ & 0.258 & 0.955 & 1.066 \\
\bottomrule
\end{tabular}

\vspace{2pt}
\begin{minipage}{\textwidth}
\footnotesize
\textit{Notes.} Results based on 200 Monte Carlo replications. True parameters:
$\lambda_1 = 1.0$, $\lambda_2 = 1.5$, $\lambda_3 = 2.0$.
Bernoulli masking with $p = 0.3$, censoring proportion $\approx 20\%$.
\end{minipage}
\end{table}

\begin{figure}[htbp]
\centering
\includegraphics[width=0.8\textwidth]{fig1_rmse_by_sample_size.pdf}
\caption{RMSE of MLE by sample size. All three component rate parameters show
decreasing RMSE as sample size increases, consistent with $\sqrt{n}$-convergence.}
\label{fig:rmse_sample_size}
\end{figure}

\begin{figure}[htbp]
\centering
\includegraphics[width=0.8\textwidth]{fig2_coverage_by_sample_size.pdf}
\caption{95\% confidence interval coverage probability by sample size. Coverage
is near the nominal 95\% level across all parameters and sample sizes, validating
the asymptotic normal approximation.}
\label{fig:coverage_sample_size}
\end{figure}

Key findings from Study 1:
\begin{enumerate}
    \item \textbf{Consistency:} Bias is small relative to RMSE at all sample
        sizes, indicating approximate unbiasedness.
    \item \textbf{Convergence:} RMSE decreases from approximately 0.5 at $n=50$
        to 0.2--0.3 at $n=200$, consistent with $\sqrt{n}$-rate convergence.
    \item \textbf{Coverage:} 95\% CI coverage ranges from 92.0\% to 96.5\%,
        close to the nominal level, validating the Fisher information-based
        standard errors.
    \item \textbf{Component effects:} Components with higher true rates
        ($\lambda_3 = 2.0$) have slightly larger absolute RMSE but similar
        relative performance.
\end{enumerate}

\subsection{Study 2: Misspecification Bias Analysis}
\label{sec:sim-bias}

We quantify the bias from incorrectly assuming C1-C2-C3 when masking is
actually informative. Data is generated with rank-based informative masking
(informativeness parameter $\alpha$), then analyzed using both the correct
model and the misspecified C2 model.

\subsubsection{Results}

Table~\ref{tab:misspecification_bias} compares bias under correct versus
misspecified models.

\begin{table}[htbp]
\centering
\caption{Bias Comparison: Correct vs Misspecified Model}
\label{tab:misspecification_bias}
\begin{tabular}{ccccc}
\toprule
$\alpha$ & Parameter & Bias (Correct) & Bias (Misspec.) & RMSE Ratio \\
\midrule
0 & $\lambda_1$ & $-0.129$ & $-0.001$ & 1.019 \\
  & $\lambda_2$ & 0.027 & 0.024 & 1.090 \\
  & $\lambda_3$ & 0.105 & $-0.020$ & 1.013 \\
\midrule
1 & $\lambda_1$ & $-0.136$ & $-0.095$ & 0.999 \\
  & $\lambda_2$ & $-0.009$ & $-0.004$ & 1.031 \\
  & $\lambda_3$ & 0.192 & 0.146 & 0.944 \\
\midrule
5 & $\lambda_1$ & $-0.129$ & $-0.127$ & 1.002 \\
  & $\lambda_2$ & $-0.005$ & 0.010 & 1.031 \\
  & $\lambda_3$ & 0.147 & 0.130 & 0.988 \\
\midrule
10 & $\lambda_1$ & $-0.153$ & $-0.152$ & 1.022 \\
   & $\lambda_2$ & 0.000 & 0.002 & 0.993 \\
   & $\lambda_3$ & 0.183 & 0.178 & 1.008 \\
\bottomrule
\end{tabular}

\vspace{2pt}
\begin{minipage}{\textwidth}
\footnotesize
\textit{Notes.} $\alpha = 0$ corresponds to non-informative masking (C2 satisfied).
As $\alpha$ increases, masking becomes more informative.
RMSE Ratio = RMSE(Misspecified) / RMSE(Correct); values $> 1$ indicate
efficiency loss.
\end{minipage}
\end{table}

\begin{figure}[htbp]
\centering
\includegraphics[width=0.9\textwidth]{fig3_misspecification_bias.pdf}
\caption{Bias comparison between correct and misspecified models as masking
informativeness increases. The misspecified model (incorrectly assuming C2)
shows similar bias patterns to the correct model for moderate informativeness.}
\label{fig:misspec_bias}
\end{figure}

\begin{figure}[htbp]
\centering
\includegraphics[width=0.8\textwidth]{fig4_rmse_ratio.pdf}
\caption{RMSE ratio (misspecified/correct) by informativeness parameter. Values
near 1 indicate minimal efficiency loss from misspecification. The maximum
ratio of 1.09 suggests the C2 assumption is reasonably robust.}
\label{fig:rmse_ratio}
\end{figure}

Key findings from Study 2:
\begin{enumerate}
    \item \textbf{Moderate robustness:} The RMSE ratio stays between 0.94 and
        1.09 across all informativeness levels, indicating that misspecifying
        the masking model produces at most 9\% efficiency loss.
    \item \textbf{Bias similarity:} Surprisingly, bias under the misspecified
        model closely tracks bias under the correct model, suggesting the C2
        assumption is more robust than theoretical arguments might suggest.
    \item \textbf{Parameter-specific effects:} Component 3 ($\lambda_3$) shows
        consistently positive bias under both models, likely due to its higher
        failure rate making it more frequently the true cause of failure.
\end{enumerate}

\subsection{Study 3: Identifiability and Candidate Set Correlation}
\label{sec:sim-ident}

We investigate how correlation between candidate set indicators affects
identifiability by examining the Fisher Information Matrix (FIM) eigenvalues.

\subsubsection{Results}

Table~\ref{tab:identifiability} presents FIM analysis by correlation level.

\begin{table}[htbp]
\centering
\caption{Fisher Information Matrix Analysis by Candidate Set Correlation}
\label{tab:identifiability}
\begin{tabular}{ccc}
\toprule
$\rho$ & Smallest Eigenvalue & Condition Number \\
\midrule
0.0 & 12.23 & 2.18 \\
0.1 & 12.93 & 2.12 \\
0.3 & 14.17 & 2.01 \\
0.5 & 15.00 & 1.97 \\
0.6 & 15.12 & 1.91 \\
0.8 & 14.33 & 2.01 \\
0.9 & 13.88 & 2.04 \\
\bottomrule
\end{tabular}

\vspace{2pt}
\begin{minipage}{\textwidth}
\footnotesize
\textit{Notes.} $\rho$ measures correlation between candidate set indicators.
As $\rho \to 1$, components always co-occur in candidate sets, theoretically
leading to non-identifiability.
\end{minipage}
\end{table}

\begin{figure}[htbp]
\centering
\includegraphics[width=0.8\textwidth]{fig5_fim_eigenvalue.pdf}
\caption{Smallest FIM eigenvalue versus candidate set correlation. The eigenvalue
remains bounded away from zero even at $\rho = 0.9$, indicating identifiability
is maintained in our simulation setup.}
\label{fig:fim_eigenvalue}
\end{figure}

\begin{figure}[htbp]
\centering
\includegraphics[width=0.8\textwidth]{fig6_rmse_by_correlation.pdf}
\caption{RMSE of MLE by candidate set correlation. Performance remains stable
across correlation levels, consistent with the FIM eigenvalue analysis.}
\label{fig:rmse_correlation}
\end{figure}

Key findings from Study 3:
\begin{enumerate}
    \item \textbf{Identifiability preserved:} The smallest FIM eigenvalue
        remains substantially positive (12--15) across all correlation levels,
        indicating parameters remain identifiable.
    \item \textbf{Condition number stable:} The condition number stays below 2.2,
        indicating a well-conditioned estimation problem.
    \item \textbf{Nonmonotonic pattern:} Interestingly, the smallest eigenvalue
        peaks around $\rho = 0.5$--$0.6$, suggesting moderate correlation may
        actually improve information content.
\end{enumerate}

\subsection{Study 4: C3 Misspecification Bias Analysis}
\label{sec:sim-c3}

We now quantify the bias from incorrectly assuming C1-C2-C3 when masking is
actually parameter-dependent (C3 violated). Data is generated with
power-weighted masking (\Cref{def:power-masking}) with varying
informativeness $\alpha$, then analyzed using both the correct model and the
misspecified C1-C2-C3 model. This study is motivated by the theoretical
result in \Cref{thm:misspec-C3}.

\subsubsection{Design}

Data is generated under the power-weighted masking model with
$\text{base\_p} = 0.5$ and $\alpha \in \{0, 0.5, 1, 2\}$, using $n = 200$
and $B = 200$ replications. Three comparisons are made:
\begin{description}
    \item[Scenario 6:] Relaxed C3 data analyzed with C1-C2-C3 model
        (misspecified).
    \item[Scenario 6b:] Relaxed C3 data analyzed with relaxed C3 model
        using known $\alpha$ (correctly specified).
    \item[Scenario 5:] C1-C2-C3 data analyzed with relaxed C3 model
        (overfitting check).
\end{description}

\subsubsection{Results}

Table~\ref{tab:c3_misspec_bias} compares bias under correct versus
misspecified models for C3 violations.

\begin{table}[htbp]
\centering
\caption{C3 Misspecification: Bias Comparison by Power Parameter $\alpha$}
\label{tab:c3_misspec_bias}
\begin{tabular}{ccccc}
\toprule
$\alpha$ & Parameter & Bias (Correct) & Bias (Misspec.) & RMSE Ratio \\
\midrule
0 & $\lambda_1$ & $-0.011$ & $-0.011$ & $1.000$ \\
  & $\lambda_2$ & $0.036$ & $0.036$ & $1.000$ \\
  & $\lambda_3$ & $0.029$ & $0.029$ & $1.000$ \\
\midrule
0.5 & $\lambda_1$ & $-0.762$ & $-0.264$ & $0.424$ \\
    & $\lambda_2$ & $0.195$ & $0.003$ & $0.801$ \\
    & $\lambda_3$ & $0.619$ & $0.314$ & $0.626$ \\
\midrule
1 & $\lambda_1$ & $-0.691$ & $-0.366$ & $0.574$ \\
  & $\lambda_2$ & $0.116$ & $-0.067$ & $1.077$ \\
  & $\lambda_3$ & $0.628$ & $0.486$ & $0.834$ \\
\midrule
2 & $\lambda_1$ & $-0.569$ & $-0.432$ & $0.788$ \\
  & $\lambda_2$ & $0.026$ & $-0.205$ & $1.850$ \\
  & $\lambda_3$ & $0.597$ & $0.691$ & $1.169$ \\
\bottomrule
\end{tabular}

\vspace{2pt}
\begin{minipage}{\textwidth}
\footnotesize
\textit{Notes.} $\alpha = 0$ corresponds to parameter-independent masking
(C3 satisfied). RMSE Ratio = RMSE(Misspecified) / RMSE(Correct); values $< 1$
indicate the misspecified model has \emph{lower} RMSE due to fewer parameters.
\end{minipage}
\end{table}

\begin{figure}[htbp]
\centering
\includegraphics[width=0.9\textwidth]{fig7_c3_misspec_bias.pdf}
\caption{C3 misspecification bias comparison. The misspecified model (ignoring
parameter-dependent masking) shows different bias patterns from the correct
model, but the magnitude of bias grows with $\alpha$.}
\label{fig:c3_misspec_bias}
\end{figure}

\begin{figure}[htbp]
\centering
\includegraphics[width=0.8\textwidth]{fig8_c3_rmse_ratio.pdf}
\caption{RMSE ratio (misspecified / correct) for C3 violations. Unlike C2
misspecification (\Cref{fig:rmse_ratio}), the ratio is often below 1,
indicating the simpler misspecified model can achieve lower RMSE despite bias.}
\label{fig:c3_rmse_ratio}
\end{figure}

\subsubsection{Interpretation}

Key findings from Study 4:
\begin{enumerate}
    \item \textbf{Bias grows with $\alpha$:} As the power parameter
        increases, the C1-C2-C3 model produces increasing bias,
        confirming \Cref{thm:misspec-C3}.
    \item \textbf{Bias-variance tradeoff:} The correctly specified relaxed C3
        model often has \emph{higher} RMSE than the misspecified model
        (RMSE ratio $< 1$), because the additional masking parameters
        increase variance. This contrasts with the C2 case (Study 2)
        where RMSE ratios were close to 1.
    \item \textbf{Overfit risk (Scenario 5):} Fitting the relaxed C3 model
        to C1-C2-C3 data produces bias for $\alpha > 0$, with convergence
        rates dropping to 80--98\%, indicating overfitting when the
        extra flexibility is unnecessary.
    \item \textbf{Comparison with C2:} While C2 misspecification
        (Study 2) showed at most 9\% efficiency loss, C3
        misspecification creates a more complex picture with
        parameter-specific effects and a bias-variance tradeoff
        favoring the simpler model in many cases.
\end{enumerate}

\subsection{Study 5: Weibull Series Systems}
\label{sec:sim-weibull}

To assess whether our findings generalize beyond exponential components, we
repeat key analyses with Weibull components. We consider a 2-component system
with shapes $\mathbf{k} = (2.0, 1.5)$ and scales
$\vlambda = (3.0, 4.0)$, using $n = 200$, $\tau = 8$, and $B = 100$
replications.

\subsubsection{Scenarios}

\begin{description}
    \item[W1:] C1-C2-C3 data $\to$ C1-C2-C3 model (baseline).
    \item[W3:] Relaxed C2 data $\to$ C1-C2-C3 model (C2 misspecification).
    \item[W4:] Relaxed C2 data $\to$ relaxed C2 model (correctly specified).
    \item[W6:] Relaxed C3 data $\to$ C1-C2-C3 model (C3 misspecification).
    \item[W7:] Relaxed C3 data $\to$ relaxed C3 model (correctly specified).
\end{description}

\subsubsection{Results}

Table~\ref{tab:weibull_sims} presents the Weibull simulation results.

\begin{table}[htbp]
\centering
\caption{Weibull Simulation Results: Bias and RMSE by Scenario}
\label{tab:weibull_sims}
\begin{tabular}{clcccc}
\toprule
Scenario & Parameter & True & Bias & RMSE & Conv. \\
\midrule
W1 & $k_1$ & 2.0 & $0.007$ & 0.124 & 100\% \\
   & $\lambda_1$ & 3.0 & $-0.016$ & 0.155 & \\
   & $k_2$ & 1.5 & $0.006$ & 0.141 & \\
   & $\lambda_2$ & 4.0 & $0.003$ & 0.376 & \\
\midrule
W3 & $k_1$ & 2.0 & $-0.043$ & 0.147 & 100\% \\
   & $\lambda_1$ & 3.0 & $-0.199$ & 0.241 & \\
   & $k_2$ & 1.5 & $-0.028$ & 0.165 & \\
   & $\lambda_2$ & 4.0 & $0.757$ & 0.982 & \\
\midrule
W4 & $k_1$ & 2.0 & $-0.005$ & 0.152 & 100\% \\
   & $\lambda_1$ & 3.0 & $-0.019$ & 0.162 & \\
   & $k_2$ & 1.5 & $0.010$ & 0.146 & \\
   & $\lambda_2$ & 4.0 & $0.041$ & 0.431 & \\
\midrule
W6 & $k_1$ & 2.0 & $-0.044$ & 0.147 & 100\% \\
   & $\lambda_1$ & 3.0 & $-0.209$ & 0.250 & \\
   & $k_2$ & 1.5 & $-0.032$ & 0.170 & \\
   & $\lambda_2$ & 4.0 & $0.821$ & 1.042 & \\
\midrule
W7 & $k_1$ & 2.0 & $-0.440$ & 0.459 & 100\% \\
   & $\lambda_1$ & 3.0 & $0.146$ & 0.278 & \\
   & $k_2$ & 1.5 & $0.639$ & 0.679 & \\
   & $\lambda_2$ & 4.0 & $-0.405$ & 0.481 & \\
\bottomrule
\end{tabular}

\vspace{2pt}
\begin{minipage}{\textwidth}
\footnotesize
\textit{Notes.} True Weibull parameters: $k_1 = 2.0$, $\lambda_1 = 3.0$,
$k_2 = 1.5$, $\lambda_2 = 4.0$. $P$ matrix: $P_{12} = 0.3$, $P_{21} = 0.5$
for relaxed C2; $\alpha = 1$, base\_p $= 0.5$ for relaxed C3.
\end{minipage}
\end{table}

\begin{figure}[htbp]
\centering
\includegraphics[width=0.7\textwidth]{fig9_weibull_baseline.pdf}
\caption{Weibull baseline MLE performance (W1). Bars show bias, error bars
indicate RMSE. All parameters are estimated with small bias and reasonable
RMSE.}
\label{fig:weibull_baseline}
\end{figure}

\begin{figure}[htbp]
\centering
\includegraphics[width=0.9\textwidth]{fig10_weibull_misspec.pdf}
\caption{Weibull misspecification comparison. Left: C2 violation (W3 vs W4).
Right: C3 violation (W6 vs W7). The correctly specified C2 model (W4)
substantially reduces bias; the C3 case is more complex.}
\label{fig:weibull_misspec}
\end{figure}

\subsubsection{Interpretation}

Key findings from Study 5:
\begin{enumerate}
    \item \textbf{Weibull baseline performs well:} Under correctly specified
        C1-C2-C3 (W1), all Weibull parameters are estimated with small bias
        and reasonable RMSE, confirming the MLE framework extends to
        non-exponential components.
    \item \textbf{C2 misspecification hurts:} Ignoring informative masking
        (W3) produces substantial bias in the scale parameter $\lambda_2$
        (bias $= 0.757$), which is largely corrected by the relaxed C2
        model (W4, bias $= 0.041$). This is more pronounced than the
        exponential case, likely because Weibull scale and shape parameters
        interact with the masking weights.
    \item \textbf{C3 misspecification in Weibull:} The misspecified model (W6)
        shows large $\lambda_2$ bias ($0.821$), but the correctly specified
        relaxed C3 model (W7) performs \emph{worse}, with substantial
        bias in the shape parameters ($k_1$: $-0.440$, $k_2$: $0.639$).
        This suggests that parameter-dependent masking is more difficult
        to handle with Weibull components due to the interaction between
        shape and scale in the power weights.
    \item \textbf{Exponential results generalize partially:} The qualitative
        finding that C2 misspecification produces predictable bias that can
        be corrected holds for Weibull systems. The C3 case requires further
        investigation for non-exponential distributions.
\end{enumerate}

\subsection{Summary of Simulation Results}
\label{sec:sim-summary}

Our simulation studies lead to the following conclusions:

\begin{enumerate}
    \item \textbf{MLE performs well:} Under the correctly specified C1-C2-C3
        Bernoulli masking model, the MLE achieves coverage near nominal levels
        and RMSE consistent with asymptotic efficiency (Study 1).

    \item \textbf{C2 misspecification is mild:} Misspecifying C2 (assuming
        non-informative masking when masking is informative) produces at most
        9\% efficiency loss and bias patterns similar to the correct model
        (Study 2).

    \item \textbf{Identifiability is robust:} Even with high correlation
        ($\rho = 0.9$) between candidate set indicators, parameters remain
        identifiable with stable FIM eigenvalues and condition numbers
        (Study 3).

    \item \textbf{C3 misspecification is nuanced:} Ignoring parameter-dependent
        masking (C3 violation) produces increasing bias with the power
        parameter $\alpha$, but the simpler misspecified model can have
        lower RMSE due to a bias-variance tradeoff (Study 4).

    \item \textbf{Weibull systems confirm and extend:} The framework
        generalizes to Weibull components. C2 misspecification produces
        larger bias for Weibull than exponential systems, reinforcing the
        value of relaxed models when masking is known to be informative
        (Study 5).

    \item \textbf{Practical guidance:} For sample sizes $n \geq 100$ with
        moderate masking and censoring, the C1-C2-C3 model provides reliable
        inference. Relaxed models are most beneficial when (a) masking is
        known to be informative, (b) the masking mechanism can be
        characterized, and (c) the sample size supports additional parameters.
\end{enumerate}

\begin{table}[htbp]
\centering
\caption{Summary of key simulation findings across all five studies.}
\label{tab:sim-summary}
\begin{tabular}{lccccc}
\toprule
Metric & Study 1 & Study 2 & Study 3 & Study 4 & Study 5 \\
\midrule
Components & 3 (exp) & 3 (exp) & 3 (exp) & 3 (exp) & 2 (Weibull) \\
RMSE range & 0.20--0.56 & 0.19--0.47 & 0.23--0.47 & 0.15--0.77 & 0.12--1.04 \\
Coverage range & 92--97\% & --- & --- & --- & --- \\
Max RMSE ratio & --- & 1.09 & --- & 1.85 & --- \\
Min FIM eigenvalue & --- & --- & 12.23 & --- & --- \\
\bottomrule
\end{tabular}
\end{table}

%% ============================================================================
\section{Discussion}
\label{sec:discussion}
%% ============================================================================

\subsection{When to Use Relaxed Models}
\label{sec:when-relaxed}

The theoretical and simulation results suggest the following practical
guidance for choosing between standard C1-C2-C3 models and relaxed alternatives.

\subsubsection{Use Standard C1-C2-C3 When:}

\begin{enumerate}
    \item \textbf{Masking mechanism is genuinely uninformative.} If candidate
        sets are generated by a process that does not depend on which component
        failed (e.g., random equipment availability for testing), C2 holds.

    \item \textbf{Masking probabilities are unknown.} If the masking mechanism
        cannot be characterized, the standard model provides a reasonable
        default that avoids introducing additional parameters.

    \item \textbf{Sample size is small.} Even if masking is slightly
        informative, the bias may be dominated by sampling variability for
        small $n$. The simpler model may provide more stable estimates.

    \item \textbf{Primary interest is in relative component reliability.}
        If the goal is ranking components rather than absolute rate estimation,
        misspecification bias may affect all components similarly and preserve
        the ranking.
\end{enumerate}

\subsubsection{Consider Relaxed Models When:}

\begin{enumerate}
    \item \textbf{Masking mechanism is known to be informative (C2).} If
        diagnostic procedures systematically favor certain components (e.g.,
        those that ``look bad'' at the failure time), C2 is violated and
        bias will result.

    \item \textbf{Masking depends on component parameters (C3).} If
        inclusion probabilities vary with the reliability parameters
        themselves (e.g., weaker components are more likely to appear in
        candidate sets), C3 is violated. The simulation studies
        (\Cref{sec:sim-c3}) show that the resulting bias grows with the
        degree of parameter-dependence, though the simpler C1-C2-C3 model
        may still have competitive RMSE due to a bias-variance tradeoff.

    \item \textbf{Masking probabilities can be estimated.} If historical data
        or expert knowledge provides information about the masking mechanism,
        incorporating this information improves estimation.

    \item \textbf{Sample size is large enough to support additional parameters.}
        Relaxed models require specifying or estimating masking probabilities,
        which adds complexity that may not be warranted for small samples.

    \item \textbf{Identifiability concerns are present.} As shown in
        \Cref{thm:ident-inform}, informative masking can resolve
        identifiability issues that arise under standard conditions.
\end{enumerate}

\subsection{Practical Guidance}
\label{sec:guidance}

Based on our analysis, we recommend the following workflow:

\begin{enumerate}
    \item \textbf{Assess the masking mechanism.} Before estimation, consider
        how candidate sets are generated. Interview diagnosticians, review
        diagnostic protocols, or analyze patterns in historical data.

    \item \textbf{Check for block structure.} Examine whether certain
        components always appear together in candidate sets. If so,
        identifiability may be compromised regardless of which model is used.

    \item \textbf{Perform sensitivity analysis.} Fit models under both
        C1-C2-C3 and plausible relaxed assumptions. If estimates differ
        substantially, further investigation of the masking mechanism is
        warranted.

    \item \textbf{Use simulation to assess impact.} Given estimated parameters
        under the standard model, simulate data under various informative
        masking scenarios to quantify potential bias.

    \item \textbf{Report uncertainty appropriately.} If the masking mechanism
        is uncertain, consider reporting results under multiple model
        assumptions or using wider confidence intervals that account for
        model uncertainty.
\end{enumerate}

\subsection{Limitations}
\label{sec:limitations}

Our analysis has several limitations:

\begin{enumerate}
    \item \textbf{Limited distribution scope.} While we extend the analysis
        to Weibull components (\Cref{sec:sim-weibull}), the closed-form
        Fisher information results focus on exponential systems. Other
        lifetime distributions (log-normal, gamma) may exhibit different
        misspecification patterns due to differing hazard structures.

    \item \textbf{Known masking probabilities.} Our relaxed models assume
        masking probabilities are known. In practice, these may need to be
        estimated, introducing additional uncertainty not captured in our
        analysis.

    \item \textbf{Independence assumption.} We assume masking for different
        observations is independent. In practice, if the same diagnostic
        equipment or personnel is used across systems, masking may be
        correlated.

    \item \textbf{Parametric masking models.} Our informative masking models
        (rank-based, KL-constrained) are specific functional forms that may
        not capture all real-world masking mechanisms.

    \item \textbf{Simulation scope.} The simulation studies cover a limited
        range of configurations. Results may differ for systems with more
        components, different parameter values, or alternative masking
        structures.
\end{enumerate}

\subsection{Future Directions}
\label{sec:future}

Several extensions would strengthen this work:

\begin{enumerate}
    \item \textbf{Semiparametric methods.} Develop estimation approaches that
        avoid fully specifying the masking mechanism, perhaps using
        nonparametric or empirical likelihood methods.

    \item \textbf{Model selection.} Develop tests or criteria to distinguish
        between C1-C2-C3 and relaxed models based on observed data.

    \item \textbf{Bayesian extensions.} Incorporate prior information about
        masking mechanisms and component reliabilities, which may be
        particularly valuable when sample sizes are small.

    \item \textbf{Sequential estimation.} For systems observed over time,
        develop methods that update masking probability estimates as data
        accumulates.

    \item \textbf{Additional lifetime distributions.} While our Weibull
        extension (\Cref{sec:sim-weibull}) validates the framework beyond
        exponential components, distributions with non-monotone hazards
        (e.g., log-normal) may present distinct challenges.

    \item \textbf{R package documentation.} Expand the
        \texttt{mdrelax} package with
        vignettes demonstrating practical application of these methods.
\end{enumerate}

%% ============================================================================
\section{Conclusion}
\label{sec:conclusion}
%% ============================================================================

We have developed a theoretical framework for likelihood-based inference in
series systems with masked failure data when the traditional conditions C2
(non-informative masking) and C3 (parameter-independent masking) are relaxed.
Our main contributions are:

\begin{enumerate}
    \item \textbf{Likelihood derivations.} We established the form of the
        likelihood under various relaxation scenarios, showing that the
        masking probabilities act as weights on component hazard contributions
        when C2 is violated.

    \item \textbf{Practical masking models.} We introduced rank-based
        informative masking and KL-divergence constrained models that provide
        interpretable parameterizations of non-standard masking.

    \item \textbf{Identifiability results.} We proved that informative masking
        can paradoxically \emph{improve} identifiability by breaking symmetries
        that cause non-identifiability under standard conditions.

    \item \textbf{Fisher information analysis.} We derived closed-form
        expressions for the Fisher information matrix under informative
        masking for exponential series systems, enabling efficiency comparisons.

    \item \textbf{Misspecification analysis.} We characterized the bias that
        arises from incorrectly assuming C2 or C3 when masking is actually
        informative or parameter-dependent, providing guidance on when
        relaxed models are necessary.

    \item \textbf{Weibull extension.} We demonstrated that the framework
        extends naturally to Weibull components, with simulation studies
        confirming the robustness findings observed for exponential systems
        while identifying additional challenges from shape-scale interactions.
\end{enumerate}

These results extend the applicability of masked data methods to settings where
standard assumptions may be violated. The accompanying \texttt{mdrelax} R package
provides implementation of these methods for practitioners. Directions for
future research are discussed in \Cref{sec:future}.


%% ============================================================================
%% Acknowledgments
%% ============================================================================

%% Acknowledgments will be added after peer review
%\section*{Acknowledgments}
%The author thanks ...

%% ============================================================================
%% References
%% ============================================================================

\bibliographystyle{plainnat}
\bibliography{refs}

%% ============================================================================
%% Appendix
%% ============================================================================

\appendix
%% ============================================================================
\section{Score Functions}
\label{app:scores}
%% ============================================================================

For reference, we provide the score functions under the misspecified and true
models for series systems with general component hazards.

\paragraph{Misspecified score (C1-C2-C3).}
The score under the standard model for parameter $\theta_j$ is:
\begin{equation}
    \frac{\partial \ell^{\mathrm{wrong}}}{\partial \theta_j} =
    \sum_{i=1}^n \frac{\partial \log R_j(s_i; \theta_j)}{\partial \theta_j}
    + \sum_{i: \delta_i = 1}
    \frac{\ind{j \in c_i}\, h_j'(s_i; \theta_j)}{\sum_{k \in c_i} h_k(s_i; \theta_k)},
\end{equation}
where $h_j' = \partial h_j / \partial \theta_j$.

\paragraph{True score (C1-C3 with known $\mathbf{P}$).}
Under the Bernoulli masking model with masking weights $\pi_{k,c}$:
\begin{equation}
    \frac{\partial \ell^{\mathrm{true}}}{\partial \theta_j} =
    \sum_{i=1}^n \frac{\partial \log R_j(s_i; \theta_j)}{\partial \theta_j}
    + \sum_{i: \delta_i = 1}
    \frac{\pi_{j,c_i}\,\ind{j \in c_i}\, h_j'(s_i; \theta_j)}
         {\sum_{k \in c_i} h_k(s_i; \theta_k)\,\pi_{k,c_i}}.
\end{equation}

\paragraph{Exponential specialization.}
For exponential components ($h_j = \theta_j$, $R_j = e^{-\theta_j t}$), these
simplify to:
\begin{align}
    \frac{\partial \ell^{\mathrm{wrong}}}{\partial \theta_j}
    &= -\sum_{i=1}^n s_i + \sum_{i: \delta_i = 1}
        \frac{\ind{j \in c_i}}{\sum_{k \in c_i} \theta_k}, \\
    \frac{\partial \ell^{\mathrm{true}}}{\partial \theta_j}
    &= -\sum_{i=1}^n s_i + \sum_{i: \delta_i = 1}
        \frac{\pi_{j,c_i}\,\ind{j \in c_i}}
             {\sum_{k \in c_i} \theta_k\,\pi_{k,c_i}}.
\end{align}

%% ============================================================================
\section{Non-Identifiability Evidence}
\label{sec:appendix-ident}
%% ============================================================================

To demonstrate the non-identifiability of $(\vtheta, \mathbf{P})$, we
attempted joint estimation with $n = 2000$ observations from a 2-component
exponential system with $\vtheta^* = (1.0, 2.0)$ and asymmetric
$\mathbf{P}$ ($p_{21} = 0.10$, $p_{12} = 0.90$).

\begin{center}
\begin{tabular}{lcc}
\toprule
Method & $\hat\theta_1$ & $\hat\theta_2$ \\
\midrule
Joint $(\vtheta, \mathbf{P})$ & 1.62 & 1.44 \\
Known $\mathbf{P}$ & 0.97 & 2.08 \\
True values & 1.00 & 2.00 \\
\bottomrule
\end{tabular}
\end{center}

The joint estimator consistently converges to
$\hat\vtheta \approx (1.6, 1.4)$ with
$\hat{P}_{21} \approx 0.45$ (true: 0.10), regardless of sample size. The
total hazard is well-identified: $\sum_j \hat\theta_j = 3.01$ vs.\ true
3.00. This confirms \Cref{thm:non-ident}: individual rates and masking
probabilities are confounded, but the total system hazard is identifiable.

%% ============================================================================
\section{Software}
\label{app:software}
%% ============================================================================

All methods are implemented in the \texttt{mdrelax} R package (version 1.1.0),
available at \url{https://github.com/queelius/mdrelax}. The package provides:
\begin{itemize}
    \item Data generation under C1-C2-C3 and relaxed C2 for both
        exponential and Weibull series systems.
    \item MLE with analytical score and Fisher information for all model tiers.
    \item The $\mathbf{P}$ matrix construction and masking probability
        computation functions used in this paper.
    \item 1276 unit tests verifying likelihood correctness, score--gradient
        agreement, FIM consistency, and MLE convergence properties.
\end{itemize}

The simulation sweep script (\texttt{paper/run\_sensitivity\_sweep.R})
reproduces all figures and tables in \Cref{sec:simulations}.


\end{document}
